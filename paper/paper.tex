\documentclass[review]{siamart1116}

%% ------------------------------------------------------------------
%% Code used in examples, needed to reproduce 
%% ------------------------------------------------------------------
%% Used for \set, used in an example below
\usepackage{braket,amsfonts}

%% Used in table example below
\usepackage{array}

%% Used in table and figure examples below
\usepackage[caption=false]{subfig}
%% Used for papers with subtables created with the subfig package
%\captionsetup[subtable]{position=bottom}
%\captionsetup[table]{position=bottom}

%% Used for PgfPlots example, shown in the "Figures" section below.
\usepackage{pgfplots}

%% Used for creating new theorems, remarks
\newsiamthm{claim}{Claim}
\newsiamremark{rem}{Remark}
\newsiamremark{expl}{Example}
\newsiamremark{hypothesis}{Hypothesis}
\crefname{hypothesis}{Hypothesis}{Hypotheses}
\usepackage{enumerate}

%% Algorithm style, could alternatively use algpseudocode
\usepackage{algorithmic}

%% For figures
\usepackage{graphicx,epstopdf}


%% For referencing line numbers
\Crefname{ALC@unique}{Line}{Lines}

%% For creating math operators
\usepackage{amsopn}
\DeclareMathOperator{\Range}{Range}

%strongly recommended
\numberwithin{theorem}{section}

%% ------------------------------------------------------------------
%% Macros for in-document examples. These are not meant to reused for
%% SIAM journal papers.
%% ------------------------------------------------------------------
\usepackage{xspace}
\usepackage{bold-extra}
\usepackage[most]{tcolorbox}
\newcommand{\BibTeX}{{\scshape Bib}\TeX\xspace}
\newcounter{example}
\colorlet{texcscolor}{blue!50!black}
\colorlet{texemcolor}{red!70!black}
\colorlet{texpreamble}{red!70!black}
\colorlet{codebackground}{black!25!white!25}

% additional useful packages
\usepackage{todonotes}
%\usepackage{showlabels}
\usepackage{autonum}

\newcommand{\edit}[1]{{\color{red} #1}}

\newcommand\bs{\symbol{'134}} % print backslash in typewriter OT1/T1
\newcommand{\preamble}[2][\small]{\textcolor{texpreamble}{#1\texttt{#2 \emph{\% <- Preamble}}}}
\def \tmpheader {tmp_\jobname_header.tex}
\def \tmpabstract {tmp_\jobname_abstract.tex}

\lstdefinestyle{siamlatex}{%
  style=tcblatex,
  texcsstyle=*\color{texcscolor},
  texcsstyle=[2]\color{texemcolor},
  keywordstyle=[2]\color{texemcolor},
  moretexcs={cref,Cref,maketitle,mathcal,text,headers,email,url},
}

\tcbset{%
  colframe=black!75!white!75,
  coltitle=white,
  colback=codebackground, % bottom/left side
  colbacklower=white, % top/right side
  fonttitle=\bfseries,
  arc=0pt,outer arc=0pt,
  top=1pt,bottom=1pt,left=1mm,right=1mm,middle=1mm,boxsep=1mm,
  leftrule=0.3mm,rightrule=0.3mm,toprule=0.3mm,bottomrule=0.3mm,
  listing options={style=siamlatex}
}

\newtcblisting[use counter=example]{example}[2][]{%
  title={Example~\thetcbcounter: #2},#1}

\newtcbinputlisting[use counter=example]{\examplefile}[3][]{%
  title={Example~\thetcbcounter: #2},listing file={#3},#1}

\DeclareTotalTCBox{\code}{ v O{} }
{ %fontupper=\ttfamily\color{texemcolor},
  fontupper=\ttfamily\color{black},
  nobeforeafter,
  tcbox raise base,
  colback=codebackground,colframe=white,
  top=0pt,bottom=0pt,left=0mm,right=0mm,
  leftrule=0pt,rightrule=0pt,toprule=0mm,bottomrule=0mm,
  boxsep=0.5mm,
  #2}{#1}

% Stretch the pages
\patchcmd\newpage{\vfil}{}{}{}
\flushbottom

%% ------------------------------------------------------------------
%% End of macros for in-document examples. 
%% ------------------------------------------------------------------

%% ------------------------------------------------------------------
%% HEADING INFORMATION
%% ------------------------------------------------------------------
\begin{tcbverbatimwrite}{\tmpheader}
\title{Symplectic Model-Reduction with a Weighted Inner Product%
  \thanks{%
\funding{Babak Maboudi Afkham is supported by the SNSF under the grant number P1ELP2\_175039. Ashish Bhatt and Bernard Haasdonk gratefully acknowledge the support of DFG grant number HA5821/5-1.}} }

\author{Babak Maboudi Afkham%
  \thanks{Institute of Mathematics (MATH), School of Basic Sciences (FSB), Ecole Polytechnique F\'ed\'erale de Lausanne, 1015 Lausanne, Switzerland (\email{babak.maboudi@epfl.ch}, \email{jan.hesthaven@epfl.ch}).}%
  \and
  Ashish Bhatt%
  \thanks{University of Stuttgart, IANS, Pfaffenwaldring 57, 70569 Stuttgart, Germany (\email{[ashish.bhatt,haasdonk]@mathematik.uni-stuttgart.de}).}
  \and
  Bernard Haasdonk%
  \footnotemark[3]
  \and
  Jan S. Hesthaven%
  \footnotemark[2]
}

% Custom SIAM macro to insert headers
\headers{Symplectic Model-Reduction with a Weighted Inner Product}
{B. M. Afkham, A. Bhatt, B. Haasdonk, and J. S. Hesthaven}
\end{tcbverbatimwrite}
\input{\tmpheader}

% Optional: Set up PDF title and authors
\ifpdf
\hypersetup{ pdftitle={Guide to Using  SIAM'S \LaTeX\ Style} }
\fi

%% ------------------------------------------------------------------
%% END HEADING INFORMATION
%% ------------------------------------------------------------------

%% ------------------------------------------------------------------
%% MAIN Document
%% ------------------------------------------------------------------
\begin{document}
\maketitle

%% ------------------------------------------------------------------
%% ABSTRACT
%% ------------------------------------------------------------------
%\begin{tcbverbatimwrite}{tmp_\jobname_abstract.tex}
\begin{tcbverbatimwrite}{\tmpabstract}
\begin{abstract}
In the recent years, considerable attention has been paid to preserving structures and invariants in reduced basis methods, in order to enhance the stability and robustness of the reduced system. In the context of Hamiltonian systems, symplectic model reduction seeks to construct a reduced system that preserves the symplectic symmetry of Hamiltonian systems. However, symplectic methods are based on the standard Euclidean inner products and are not suitable for problems equipped with a more general inner product. In this paper we generalize symplectic model reduction to allow for the norms and inner products that are most appropriate to the problem while preserving the symplectic symmetry of the Hamiltonian systems. To construct a reduced basis and accelerate the evaluation of nonlinear terms, a greedy generation of a symplectic basis is proposed. Furthermore, it is shown that the greedy approach yields a norm bounded reduced basis. The accuracy and the stability of this model reduction technique is illustrated through the development of reduced models for a vibrating elastic beam and the sine-Gordon equation.
\end{abstract}

\begin{keywords}
Structure Preserving, Weighted MOR, Hamiltonian Systems, Greedy Reduced Basis, Symplectic DEIM
\end{keywords}

\begin{AMS}
78M34, 34C20, 35B30, 37K05, 65P10, 37J25
\end{AMS}
\end{tcbverbatimwrite}
\input{\tmpabstract}
%% ------------------------------------------------------------------
%% END HEADER
%% ------------------------------------------------------------------

%% ------------------------------------------------------------------
%% MAIN Body
%% ------------------------------------------------------------------

\section{Introduction}
\label{sec:intro}

Reduced order models have emerged as a powerful approach to cope with increasingly complex new applications in engineering and science. These methods substantially reduce the dimensionality of the problem by constructing a reduced configuration space. Exploration of the reduced space is then possible with significant acceleration \cite{hesthaven2015certified,Haasdonk2017}.

Over the past decade, reduced basis (RB) methods have demonstrated great success in lowering of the computational costs of solving elliptic and parabolic differential equations \cite{ito1998reduced,ito2001reduced}. However, model order reduction (MOR) of hyperbolic problems remains a challenge. Such problems often arise from a set of conservation laws and invariants. These intrinsic structures are lost during MOR which results in a qualitatively wrong, and sometimes unstable reduced system \cite{Amsallem:2014ef}.

%To have a sense of this error, error estimation is important from applications point of view \cite{HaasdonkOhlberger11,RuinerEtAl12,BhattEtAl18}. But it can difficult and expensive to compute useful error bounds. When one is interested in a cheap surrogate for the error incurred or when the conserved quantity is an output of the system, it becomes imperative to preserve this structure through model order reduction.

Recently, the construction of RB methods that conserve intrinsic structures has attracted attention \cite{doi:10.1137/17M1111991,1705.00498,kalashnikova2014stabilization,farhat2015structure,doi:10.1137/110836742,doi:10.1137/140959602,beattie2011structure,doi:10.1137/140978922}. Structure preservation in MOR not only constructs a physically meaningful reduced system, but can also enhance the robustness and stability of the reduced system. In system theory, conservation of passivity can be found in the work of \cite{polyuga2010structure,gugercin2012structure}. Energy preserving and inf-sup stable methods for finite element methods (FEM) are developed in \cite{farhat2015structure,ballarin2015supremizer}. Also, a conservative MOR technique for finite-volume methods is proposed in \cite{1711.11550}.

Moreover, the simulation of reduced models incurs solution errors and the estimation of this error is essential in applications of MOR \cite{HaasdonkOhlberger11,RuinerEtAl12,BhattEtAl18}. Finding tight error bounds for a general reduced system has shown to be computationally expensive and often impractical. Therefore, when one is interested in a cheap surrogate for the error or when the conserved quantity is an output of the system, it becomes imperative to preserve system structures in the reduced model.

In the context of Lagrangian and Hamiltonian systems, recent works provide a promising approach to the construction of robust and stable reduced systems. Carlberg, Tuminaro, and Boggs \cite{doi:10.1137/140959602} suggest that a reduced order model of a Lagrangian system be identified by an approximate Lagrangian on a reduced order configuration space. This allows the reduced system to inherit the geometric structure of the original system. A similar approach has been adopted in the work of Peng and Mohseni \cite{doi:10.1137/140978922} and in the work of Maboudi Afkham and Hesthaven \cite{doi:10.1137/17M1111991} for Hamiltonian systems. They construct a low-order symplectic linear vector space, i.e. a vector space equipped with a symplectic 2-form, as the reduced space. Once the symplectic reduced space is generated, a symplectic projection result in a physically meaningful reduced system. A proper time-stepping scheme then preserves the Hamiltonian structure of the reduced system. It is shown in \cite{doi:10.1137/17M1111991,doi:10.1137/140978922} that this approach preserves the overall dynamics of the original system and enhances the stability of the reduced system. Despite the success of these method in MOR of Hamiltonian systems, these techniques are only compatible with the Euclidean inner product. Therefore, the computational structures that arise from a natural inner product of a problem will be lost during MOR.

Weak formulations and inner-products, defined on a Hilbert space, are at the core of the error analysis of many numerical methods for solving partial differential equations. Therefore, it is natural to seek MOR methods that consider such features. At the discrete level, these features often require a Euclidean vector space to be equipped with a generalized inner product, associated with a weight matrix $X$. Many works enabled conventional MOR techniques compatible with such inner products \cite{sen2006natural}. However, a MOR method that simultaneously preserves the symplectic symmetry of Hamiltonian systems remains unknown. 

In this paper, we seek to combine a classical MOR method with respect to a weight matrix with the symplectic MOR. Generalized inner products often require a state transformation to a non-canonical coordinate system. This restricts the choices of symplectic integrators that can be used in a numerical integration of the reduced system. The reduced system constructed by the new method, however, is a Hamiltonian system on a symplectic linear vector space with a canonical coordinate system. Therefore, any symplectic integrator can ensure long-time and robust numerical integration. It is demonstrated that the new method can be viewed as the natural extension to \cite{doi:10.1137/17M1111991}, and therefore retains the structure preserving features, e.g. symplecticity and stability. We also present a greedy approach for the construction of a generalized symplectic basis for the reduced system. Symplectic bases are often non-orthogonal and therefore not norm bounded. However, we show that the condition number of the basis generated by the greedy method is bounded by the condition number of the weight matrix $X$. Finally, to accelerate the evaluation of nonlinear terms in the reduced system, we present a variation of the discrete empirical interpolation method (DEIM) that preserves the symplectic structure of the reduced system.

What remains of this paper is organized as follows. In \cref{sec:hamil} we cover the required background on the Hamiltonian and the generalized Hamiltonian systems. \Cref{sec:mor} summarizes classic MOR routine with respect to a weighted norm and the symplectic MOR method with respect to the standard Euclidean inner product. We introduce the symplectic MOR method with respect to a weighted inner product in \cref{sec:normmor}. \Cref{sec:res} illustrates the performance of the new method through a vibrating beam and the sine-Gordon equation. We offer a few conclusive remarks in \cref{sec:conc}.

\section{Hamiltonian systems}
\label{sec:hamil}

Let $(\mathcal Z, \Omega)$ be a symplectic linear vector space \cite{Marsden:2010:IMS:1965128}, with $\mathcal Z \subset \mathbb R^{2n}$ the configuration space and $\Omega:\mathbb R^{2n}\times\mathbb R^{2n} \to \mathbb R$ a closed, skew-symmetric and non-degenerate 2-form on $\mathcal Z$. Given a smooth function $H:\mathbb R^{2n} \to \mathbb R$, the so called \emph{Hamiltonian}, the \emph{Hamiltonian system} of evolution reads
\begin{equation} \label{eq:hamil.1}
\left\{
\begin{aligned}
	& \dot z = \mathbb J_{2n} \nabla_z H(z),  \\
	&  z(0) = z_0.
\end{aligned}
\right.
\end{equation}
Here $z\in \mathcal Z$ \edit{is the state vector} and $\mathbb J_{2n}$ is the \emph{symplectic} or \emph{canonical} matrix
\begin{equation} \label{eq:hamil.2}
	\mathbb{J}_{2n} = 
	\begin{pmatrix}
	0_n & I_n \\
	-I_n & 0_n
	\end{pmatrix},
\end{equation}
such that $\Omega(x,y) = x^T\mathbb J_{2n}y$, for all state vectors $x,y\in \mathbb R^{2n}$ \cite{Marsden:2010:IMS:1965128}. Here $0_n$ and $I_n$ are the zero matrix and the identity matrix of size $n\times n$, respectively.

A general coordinate transformation does not in general preserve canonical properties of a Hamiltonian system (\ref{eq:hamil.1}). Indeed only transformations which preserve the symplectic form, \emph{symplectic transformations}, preserve the form of a Hamiltonian system \cite{Hairer:1250576}.

Suppose that $(\mathcal Z,\Omega)$ and $(\mathcal Y,\Lambda)$, with $\mathcal Z \subset \mathbb R^{2n}$ and $\mathcal Y \subset \mathbb R^{2n}$, are two symplectic linear vector spaces with a canonical basis. A transformation $\mu:\mathcal Z\to\mathcal Y$ is symplectic if
\begin{equation} \label{eq:hamil.3}
	\Omega(x,y) = \Lambda(\mu_z(z)x,\mu_z(z)y), \quad \text{for all } x,y\in\mathcal Z,
\end{equation}
where subscript $z$ denotes the gradient. Therefore symplectic 2-forms of a pair of vectors and their images under a symplectic transformation are equal. It is straightforward to see that a linear transformation $\mu(x) = Ax$, with $A\in \mathbb R^{2n\times 2k}$ and $\Omega = \Lambda$, is symplectic if
\begin{equation} \label{eq:hamil.4}
	A^T \mathbb J_{2n} A = \mathbb J_{2k}.
\end{equation}

%We are interested in a class of symplectic transformations that transform a symplectic structure $J_{2n}$ into the standard symplectic structure $\mathbb J_{2k}$.
%\begin{definition} \label{def:symp-mat}
%Let $J_{2n}\in \mathbb R^{2n\times 2n}$ be a full-rank skew-symmetric structure matrix. A matrix $A\in\mathbb R^{2n\times 2k}$ is $J_{2n}$-symplectic if
%\begin{equation} \label{eq:hamil.5}
%A^T J_{2n} A = \mathbb{J}_{2k}.
%\end{equation}
%\end{definition}
%Note that in the literature \cite{Marsden:2010:IMS:1965128,Hairer:1250576}, symplectic transformations refer to $\mathbb{J}_{2n}$-symplectic matrices, in contrast to \Cref{def:symp-mat}.

A central feature of Hamiltonian systems is preservation of the Hamiltonian and the symplectic form by the flow of the system.
\begin{theorem} \label{thm:1}
\cite{Marsden:2010:IMS:1965128} The Hamiltonian $H$ is a conserved quantity of the Hamiltonian system \eqref{eq:hamil.1} i.e. $H(z(t)) = H(z_0)$ for all $t \geq 0$. Moreover, the flow $\phi_{t,H}:z_0 \to z(t;z_0)$ of a Hamiltonian system is symplectic, i.e.,
$$\partial_z \phi_{t,H}(z)^T \mathbb J_{2n} \partial_z \phi_{t,H}(z) = \mathbb J_{2n}.$$
\end{theorem}

Preservation of phase space area or volume is a consequence of both, a Hamiltonian system with the standard symplectic structure and a symplectic transformation. Therefore, it is natural to expect a numerical integrator that solves (\ref{eq:hamil.1}) to also satisfy the conservation laws expressed in  \Cref{thm:1}. Conventional numerical time integrators, e.g. general Runge-Kutta methods, do not generally preserve the symplectic symmetry of Hamiltonian systems which often result in an unphysical behavior of the solution over long time-integration. The \emph{St\"ormer-Verlet} scheme is an example of a second order symplectic time-integrator given as
\begin{equation} \label{eq:hamil.6}
	\begin{aligned}
	q_{m+1/2} &= q_m + \frac{\Delta t} 2 ~ \nabla_p \tilde H(p_m,q_{m+1/2}), \\
	p_{m+1} &= p_m - \frac{\Delta t} 2  ~ \left( \nabla_q \tilde H(p_m,q_{m+1/2}) + \nabla_{q} \tilde H(p_{m+1},q_{m+1/2}) \right), \\
	q_{m+1} &= q_{m+1/2} + \frac{\Delta t} 2  ~  \nabla_p \tilde H(p_{m+1},q_{m+1/2}).
	\end{aligned}
\end{equation}
Here, $\tilde z = (q^T,p^T)^T$, $\tilde H(\tilde z) = H(\mathcal T^{-1}z)$, $\Delta t$ denotes a uniform time step-size, and $q_m \approx q(m\Delta t)$ and $p_m \approx p(m\Delta t)$, $m \in \mathbb{N} \cup \{ 0\}$, are approximate numerical solutions. For \cref{eq:hamil.1} is an example of such integrators.

Note that \Cref{thm:1} is also valid for a Hamiltonian system in a transformed coordinate system, associated with a skew-symmetric and full rank structure matrix $J_{2n}$. Such Hamiltonian systems also carry symmetries, e.g., the symmetry expressed in \Cref{thm:1} or the preservation of the phase space volume \cite{Hairer:1250576}. However, to ensure a robust and long time-integration, geometric numerical integration of Hamiltonian systems that exploits such symmetries is preferred and better established in a canonical coordinate systems \cite{Hairer:1250576,bhatt2017structure}.  For more on the construction and the applications of symplectic and geometric numerical integrators, we refer the reader to \cite{Hairer:1250576,bhatt2017structure}.

\section{Model order reduction}
\label{sec:mor}

In this section we summarize the fundamentals of MOR and discuss the conventional approach to MOR with a weighted inner product. We then recall the main results from \cite{doi:10.1137/17M1111991} regarding symplectic MOR. In \cref{sec:normmor} we shall combine the two concepts to introduce the symplectic MOR of Hamiltonian systems with respect to a weighted inner product.

\subsection{Model-reduction with a weighted inner product} \label{sec:mor.1}
Consider a dynamical system of the form
\begin{equation} \label{eq:mor.1}
\left\{
\begin{aligned}
	\dot x(t) &= f(t,x), \\
	x(0) &= x_0.
\end{aligned}
\right.
\end{equation}
where $x\in \mathbb R^{m}$ and $f:\mathbb R \times \mathbb R^{m} \to \mathbb R^{m}$ is some continuous function. In this paper we assume that the time $t$ is the only parameter on which the solution vector $x$ depends. Nevertheless, it is straightforward to generalize the findings of this paper to the case of parametric MOR, where $x$ depends on a larger set of parameters that belong to a closed and bounded subset.

Suppose that $x$ is well approximated by a low dimensional linear subspace with the basis matrix $V=[v_1|\dots|v_k]\in \mathbb R^{m\times k}$, $v_i\in \mathbb R^{m}$ for $i=1,\dots,k$. The approximate solution to (\ref{eq:mor.1}) in this basis reads
\begin{equation} \label{eq:mor.2}
	x \approx Vy,
\end{equation}
where $y \in \mathbb R^k$ are the expansion coefficients of $x$ in the basis $V$. Note that projection of $x$ onto colspan$(V)$ depends on the inner product and the norm defined on (\ref{eq:mor.1}). We define the weighted inner product
\begin{equation} \label{eq:mor.3}
	\left\langle x,y \right\rangle_X = x^TXy,\quad \text{for all } x,y \in \mathbb R^m,
\end{equation}
for some symmetric and positive-definite matrix $X\in \mathbb{R}^{m\times m}$ and refer to $\|\cdot \|_X$ as the $X$-norm associated to this inner product. If we choose $V$ to be an orthonormal basis with respect to the $X$-norm ($V^TXV=I_k$), then the operator
\begin{equation} \label{eq:mor.4}
	P_{X,V}(x) = VV^TXx, \quad \text{for all } x\in \mathbb R^{m}
\end{equation}
becomes idempotent, i.e. $P_{X,V}$ is a projection operator onto colspan$(V)$.

Now suppose that the \emph{snapshot matrix} $S=[x(t_1)|x(t_2)|\ldots|x(t_N)]$ is a collection of $N$ solutions to (\ref{eq:mor.1}) at time instances $t_1,\dots,t_N$. We seek $V$ such that it minimizes the collective projection error of the samples onto colspan$(V)$ which corresponds to the minimization problem
\begin{equation} \label{eq:mor.5}
\begin{aligned}
& \underset{V\in \mathbb{R}^{m\times k}}{\text{minimize}}
& & \sum_{i=1}^N \| x(t_i) - P_{X,V}( x(t_i) ) \|_X^2, \\
& \text{subject to}
& & V^TXV = I_k.
\end{aligned}
\end{equation}
Note that the solution to (\ref{eq:mor.5}) is known as the proper orthogonal decomposition (POD) \cite{hesthaven2015certified,quarteroni2015reduced,gubisch2017proper}. Following \cite{quarteroni2015reduced} the above minimization is equivalent to
\begin{equation} \label{eq:mor.6}
\begin{aligned}
& \underset{\tilde V\in \mathbb{R}^{m\times k}}{\text{minimize}}
& & \| \tilde S - \tilde V \tilde V^T \tilde S \|_F^2, \\
& \text{subject to}
& & \tilde V^T\tilde V = I_k.
\end{aligned}
\end{equation}
where $\tilde V = X^{1/2} V$, $\tilde S = X^{1/2} S$, and $X^{1/2}$ is the matrix square root of $X$. According to the Schmidt-Mirsky-Eckart-Young theorem \cite{Markovsky:2011:LRA:2103589} the solution $\tilde V$ to the minimization (\ref{eq:mor.6}) is the truncated singular value decomposition (SVD) of $\tilde S$. The basis $V$ then is $V = X^{-1/2}\tilde V$. The reduced model of (\ref{eq:mor.1}), using the basis $V$ and the projection $P_{X,V}$, is
\begin{equation} \label{eq:mor.7}
	\left\{
	\begin{aligned}
	\dot y(t) &= V^TX f(t,Vy), \\
	y(0) &= V^TX x_0.
	\end{aligned}
	\right.
\end{equation}
If $k$ can be chosen such that $k \ll m$, then the reduced system (\ref{eq:mor.7}) can potentially be evaluated significantly faster than the full order system (\ref{eq:mor.1}). Finding the matrix square root of $X$ can often be computationally exhaustive. In such cases, explicit use of $X^{1/2}$ can be avoided by finding the eigen-decomposition of the \emph{Gramian} matrix $G = S^TXS$ \cite{quarteroni2015reduced,Haasdonk2017}.

Besides RB methods, there exist other ways of basis generation e.g. greedy strategies, the Krylov subspace method, balanced truncation, Hankel-norm approximation etc. \cite{antoulas2005approximation}. We refer the reader to \cite{hesthaven2015certified,quarteroni2015reduced,Haasdonk2017} for further information regarding the development and the efficiency of reduced order models. 

\subsection{Symplectic MOR} \label{sec:mor.2}
Hamiltonian systems are a special case of port-Hamiltonian systems where energy of the system is bounded by the work done on the system instead of being constant. Structure-preserving reduction methods , based on POD and optimal subspaces, and hyper-reduction methods for nonlinear port-Hamiltonian systems are introduced in \cite{chaturantabut2016structure}, where they also derive a priori error bounds. The present work has similar intentions and builds on \cite{doi:10.1137/140978922,doi:10.1137/17M1111991} to hyper-reduce a Hamiltonian system minimizing a weighted norm, which could be more attractive in certain applications.

Conventional MOR methods, e.g. those introduced in \cref{sec:mor.1}, do no generally preserve the conservation laws expressed in \cref{thm:1}. As mentioned earlier, this often results in the lack of robustness in the reduced system over long time-integration. In this section we summarize the main findings of \cite{doi:10.1137/17M1111991} regarding symplectic model order reduction of Hamiltonian systems with respect to the standard Euclidean inner product. Symplectic MOR aims to construct a reduced system that conserves the geometric symmetry expressed in \Cref{thm:1} which helps with the stability of the reduced system.
Consider a Hamiltonian system \cref{eq:hamil.1} with the standard structure matrix
\begin{equation} \label{eq:mor.8}
\left\{
\begin{aligned}
	\dot z(t) &= \mathbb J_{2n} \nabla_z H(z), \\
	z(0) &= z_0.
\end{aligned}
\right.
\end{equation}
Here $z\in \mathbb R^{2n}$ is the state vector and $f:\mathbb R^{2n}\to\mathbb R$ is sufficiently smooth function. Suppose that the solution to (\ref{eq:mor.8}) is well approximated by a low dimensional symplectic subspace. Let $A\in \mathbb{R}^{2n\times 2k}$ be a $\mathbb{J}_{2n}$-symplectic basis containing the basis vectors $A=[e_1|\dots|e_k|f_1|\dots|f_k]$, such that $z \approx Ay$ with $y \in \mathbb{R}^{2k}$ the expansion coefficients of $z$ in this basis. Using the symplectic inverse $A^+ := \mathbb J_{2k}^T A^T \mathbb J_{2n}$ we can construct the reduced system
\begin{equation} \label{eq:mor.9}
	\dot y = \mathbb J_{2n} (A^+)^T \nabla_y H(Ay).
\end{equation}
We refer the reader to \cite{doi:10.1137/17M1111991} for the details of the derivation. It is shown in \cite{doi:10.1137/140978922} that $(A^+)^T$ is also $\mathbb J_{2n}$-symplectic, therefore $A^+ \mathbb J_{2n} (A^+)^T = \mathbb J_{2k}$ and (\ref{eq:mor.9}) reduces to the Hamiltonian system
\begin{equation} \label{eq:mor.10}
	\dot y(t) = \mathbb J_{2k} \nabla_y H(Ay)
\end{equation}
with the Hamiltonian $\mathcal H(y) = H(Ay)$.

To reduce the complexity of evaluating the nonlinear term in (\ref{eq:mor.10}), we may apply the discrete empirical interpolation method (DEIM) \cite{barrault2004empirical,Chaturantabut:2010cz,wirtz2014posteriori}. Assuming that $\nabla_z H(z)$ lies near a low dimensional subspace with a basis matrix $U\in \mathbb R^{2n\times r}$ the DEIM approximation reads
\begin{equation} \label{eq:mor.11}
	\nabla_z f(z) \approx U (\mathcal P^T U)^{-1} \mathcal P^T \nabla_z H(z).
\end{equation}
Here $\mathcal P \in \mathbb R^{2n\times r}$ is the interpolating index matrix \cite{Chaturantabut:2010cz}. For a general choice of $U$ the approximation in (\ref{eq:mor.11}) destroys the Hamiltonian structure, if inserted in (\ref{eq:mor.8}). It is shown in \cite{doi:10.1137/17M1111991} that by taking $U = (A^+)^T$ we can recover the Hamiltonian structure in (\ref{eq:mor.10}). Therefore, the reduced system to (\ref{eq:mor.8}) becomes
\begin{equation} \label{eq:mor.12}
\left\{
\begin{aligned}
	\dot y(t) &= \mathbb J_{2k} (A^+)^T(\mathcal P^T (A^+)^T)^{-1} \mathcal P^T \nabla_z H(Ay), \\
	y(0) &= A^+ z_0.
\end{aligned}
\right.
\end{equation}
Note that the Hamiltonian formulation of (\ref{eq:mor.12}) allows us to integrate it using a symplectic integrator. This conserves the symmetry expressed in \Cref{thm:1} at the level of the reduced system. It is also shown in \cite{doi:10.1137/17M1111991,doi:10.1137/140978922} that the stability of the critical points of (\ref{eq:mor.8}) is preserved in the reduced system and the difference of the Hamiltonians of the two system \cref{eq:mor.8,eq:mor.12} is constant. Therefore, the overall behavior (\ref{eq:mor.12}) is close to the full order Hamiltonian system (\ref{eq:mor.8}). In the next subsection we discuss methods for generating a $\mathbb J_{2n}$-symplectic basis $A$.

\subsection{Greedy generation of a $\mathbb J_{2n}$-symplectic basis} \label{sec:mor.3}
Suppose that $S \in \mathbb R^{2n\times N}$ is the snapshot matrix containing the time instances $\{z(t_i)\}_{i=1}^N$ of the solution to (\ref{eq:mor.8}). We seek the $\mathbb J_{2n}$-symplectic basis $A$ such that the collective symplectic projection error of samples in $S$ onto colspan$(A)$ is minimized.
\begin{equation} \label{eq:mor.13}
\begin{aligned}
& \underset{A\in \mathbb{R}^{2n\times 2k}}{\text{minimize}}
& & \| S - P^\text{symp}_{I,A}(S) \|_F^2, \\
& \text{subject to}
& & A^T\mathbb J_{2n}A = \mathbb J_{2k}.
\end{aligned}
\end{equation}
Here $P^\text{symp}_{I,A} = AA^+$ is the symplectic projection operator with respect to the standard Euclidean inner product onto colspan$(A)$. Note that $P^\text{symp}_{I,A} \circ P^\text{symp}_{I,A} = P^\text{symp}_{I,A}$ \cite{doi:10.1137/140978922,doi:10.1137/17M1111991}.

Direct approaches to solve (\ref{eq:mor.13}) are often inefficient. Some SVD-type solutions to (\ref{eq:mor.13}) are proposed by \cite{doi:10.1137/140978922}. However, the form of the suggested basis, e.g. the block diagonal form suggested in \cite{doi:10.1137/140978922}, is not compatible with a general weight matrix $X$. 

The greedy generation of a $\mathbb J_{2n}$-symplectic basis aims to find a near optimal solution to (\ref{eq:mor.13}) in an iterative process. This method increases the overall accuracy of the basis by adding the best possible basis vectors at each iteration. Suppose that $A_{2k} = [e_1|\dots|e_k|\mathbb J_{2n}^T e_1|\dots|\mathbb J_{2n}^T e_k]$ is a $\mathbb J_{2n}$-symplectic and orthonormal basis \cite{doi:10.1137/17M1111991}. The first step of the greedy method is to find the snapshot $z_{k+1}$, that is worst approximated by the basis $A_{2k}$:
\begin{equation} \label{eq:mor.14}
	z_{k+1} := \underset{z \in \{ z(t_i) \}_{i=1}^N}{\text{argmax } }\| z - P^\text{symp}_{I,A_{2k}}(z) \|_2. 
\end{equation}
Note that if $z_{k+1}\neq 0$ then $z_{k+1}$ is not in colspan$(A_{2k})$. Then we obtain a non-trivial vector $e_{k+1}$ by $\mathbb J_{2n}$-orthogonalizing $z_{k+1}$ with respect to $A_{2k}$:
\begin{equation} \label{eq:mor.14.1}
	\tilde z = z_{k+1} - A_{2k}\alpha, \quad e_{k+1} = \frac{\tilde z}{\|\tilde z \|_2}.
\end{equation}
Here, $\alpha\in \mathbb R^{2k}$ are the expansion coefficients of the projection of $z$ onto the column span of $A_{2k}$ where $\alpha_i = -\Omega(z_{k+1},\mathbb J_{2n}^Te_i)$ for $i\leq k$ and $\alpha_i = \Omega(z_{k+1},e_i)$ for $i>k$. Since $\Omega(e_{k+1},\mathbb{J}_{2n}^T e_{k+1}) = \| e_{k+1} \|_2^2 \neq 0$ the enriched basis $A_{2k+2}$ reads
\begin{equation} \label{eq:mor.15}
	A_{2k+2} = [e_1|\dots|e_k|e_{k+1}|\mathbb J_{2n}^Te_1|\dots|\mathbb J_{2n}^Te_{k+1}].	
\end{equation}
It is easily verified that $A_{2k+2}$ is $\mathbb J_{2n}$-symplectic and orthonormal. This enrichment continues until the given tolerance is satisfied. We note that the choice of the orthogonalization routine generally depends on the application. In this paper we use the symplectic Gram-Schmidt (GS) process as the orthogonalization routine. However the isotropic Arnoldi method or the isotropic Lanczos method \cite{doi:10.1137/S1064827500366434} are backward stable alternatives.

MOR is specially useful in reducing parametric models that depend on a closed and bounded parameter set $\mathcal{S} \subset \mathbb R^{d}$ characterizing physical properties of the underlying system. The evaluation of the projection error is impractical for such problems. The loss in the Hamiltonian function can be used as a cheap surrogate to the projection error. Suppose that a $\mathbb J_{2n}$-symplectic basis $A_{2k}$ is given, then one selects a new parameter $\omega_{k+1} \in \mathcal{S}$ by greedy approach:
\begin{equation} \label{eq:mor.16}
	\omega_{k+1} = \underset{\omega \in \mathcal{S}}{\text{argmax } } | H(z(\omega)) - H(P^\text{symp}_{I,A}(z(\omega))) |,
\end{equation}
and then enriches the basis $A_{2k}$ as discussed above. It is shown in \cite{doi:10.1137/17M1111991} that the loss in the Hamiltonian is constant in time. Therefore, $\omega_{k+1}$ can be identified in the \emph{offline phase} before simulating the reduced order model. Note that the relation between the projection error \cref{eq:mor.14} and the error in the Hamiltonian \cref{eq:mor.16} is still unknown.

We summarize the greedy algorithm for generating a $\mathbb J_{2n}$-symplectic basis in \Cref{alg:1}. The first loop constructs a $\mathbb J_{2n}$-symplectic basis for the Hamiltonian system (\ref{eq:mor.8}), and the second loop adds the nonlinear snapshots to the symplectic inverse of the basis. We refer the reader to \cite{doi:10.1137/17M1111991} for more details. In \cref{sec:normmor} we will show how this algorithm can be generalized to support any weighted inner product.

\begin{algorithm} 
\caption{The greedy algorithm for generation of a $\mathbb J_{2n}$-symplectic basis} \label{alg:1}
{\bf Input:} Tolerated projection error $\delta$, initial condition $ z_0$, snapshots $\mathcal Z = \{ z(t_i) \}_{i=1}^{N}$ and $\mathcal G = \{ \nabla_z H(z(t_i)) \}_{i=1}^{N}$
\begin{enumerate}
%\item $t^1 \leftarrow t=0$
\item $e_1 \leftarrow \frac{z_0}{\|z_0\|_2}$
\item $A \leftarrow [e_1|\mathbb J^T_{2n}e_1]$
\item $k \leftarrow 1$
\item \textbf{while} $\| z - P^\text{symp}_{I,A}( z ) \|_2 > \delta$ for any $z\in \mathcal Z$
\item \hspace{0.5cm} $z_{k+1} := \underset{z\in \mathcal Z}{\text{argmax }} \| z - P^\text{symp}_{I,A}( z ) \|_2$
\item \hspace{0.5cm} $\mathbb J_{2n}$-orthogonalize $ z_{k+1}$ to obtain $e_{k+1}$
\item \hspace{0.5cm} $A \leftarrow [e_1|\dots |e_{k+1} | \mathbb J^T_{2n}e_1|\dots,\mathbb J^T_{2n}e_{k+1}]$
\item \hspace{0.5cm} $k \leftarrow k+1$
\item \textbf{end while}
\item compute $(A^+)^T=[e'_1|\dots|e'_k|\mathbb J^T_{2n}e'_1|\dots|\mathbb J^T_{2n}e'_k]$
\item \textbf{while} $\| g - P^\text{symp}_{I,(A^+)^T}(g) \|_2 > \delta$ for all $g \in \mathcal G$
\item \hspace{0.5cm} $g_{k+1} := \underset{g \in \mathcal G}{\text{argmax }} \| g - P^\text{symp}_{I,(A^+)^T}(g) \|_2$
\item \hspace{0.5cm} $\mathbb J_{2n}$-orthogonalize $g_{k+1}$ to obtain $e'_{k+1}$
\item \hspace{0.5cm} $(A^+)^T \leftarrow [e'_1|\dots |e'_{k+1} | \mathbb J^T_{2n}e'_1|\dots|\mathbb J^T_{2n}e'_{k+1}]$
\item \hspace{0.5cm} $k \leftarrow k+1$
\item \textbf{end while}
\item $A \leftarrow \left( \left( \left( A^+\right) ^T \right ) ^+ \right)^T$
\end{enumerate}
\vspace{0.5cm}
{\bf Output:} $\mathbb J_{2n}$-symplectic basis $A$.
\end{algorithm}

\section{Symplectic model-reduction with a weighted inner product} \label{sec:normmor}

In this section we combine the concept of model reduction with a weighted inner product, discussed in \Cref{sec:mor.1}, with the symplectic model reduction discussed in \Cref{sec:mor.2}. We will argue that the new method can be viewed as a natural extension of the original symplectic method. Finally we generalize the greedy method for the symplectic basis generation, and the symplectic model reduction of nonlinear terms to be compatible with any weighted inner product.

\subsection{Generalization of the symplectic projection} \label{sec:normmor.1}
As discussed in \Cref{sec:mor.1}, the error analysis of methods for solving partial differential equations often require the use of a weighted inner product. This is particularly important when dealing with Hamiltonian systems, where the system energy can induce a norm that is fundamental to the dynamics of the system.

Consider a Hamiltonian system of the form (\ref{eq:mor.8}) together with the weighted inner product defined in (\ref{eq:mor.3}) with $m=2n$. Also suppose that the solution $z$ to (\ref{eq:mor.8}) is well approximated by a $2k$ dimensional symplectic subspace with the basis $A$. We seek to construct a projection operator that minimizes the projection error with respect to the $X$-norm while preserving the symplectic dynamics of (\ref{eq:mor.8}) in the projected space. Consider the operator $P: \mathbb R^{2n} \to \mathbb R^{2n}$ defined as
\begin{equation} \label{eq:normmor.1}
	P = A \mathbb J_{2k}^T A^T X \mathbb J_{2n} X.
\end{equation}
It is easy to show that $P$ is idempotent if and only if
\begin{equation} \label{eq:normmor.2}
	\mathbb J_{2k}^T A^T X \mathbb J_{2n} X A = I_{2k}.
\end{equation}
In which case $P$ is a projection operator onto colspan$(A)$. Suppose that $S$ is the snapshot matrix containing the time samples $\{z(t_i)\}_{i=1}^N$ of the solution to (\ref{eq:mor.8}). We seek to find the basis $A$ That minimizes the collective projection error of snapshots with respect to the $X$-norm,
\begin{equation} \label{eq:normmor.3}
\begin{aligned}
& \underset{A\in \mathbb{R}^{2n\times 2k}}{\text{minimize}}
& & \sum_{i=1}^N \| z(t_i) - P(z(t_i)) \|_X^2, \\
& \text{subject to}
& & \mathbb J_{2k}^T A^T X \mathbb J_{2n} X A = I_{2k}.
\end{aligned}
\end{equation}
By (\ref{eq:normmor.1}) we have
\begin{equation} \label{eq:normmor.4}
\begin{aligned}
	\sum_{i=1}^N \| z(t_i) - P(z(t_i)) \|_X^2 &= \sum_{i=1}^N \| z(t_i) - A \mathbb J_{2k}^T A^T X \mathbb J_{2n} X z(t_i) \|_X^2 \\
	&= \sum_{i=1}^N \| X^{1/2}z(t_i) - X^{1/2} A \mathbb J_{2k}^T A^T X \mathbb J_{2n} X z(t_i) \|_2^2 \\
	&= \| X^{1/2} S - X^{1/2} A \mathbb J_{2k}^T A^T X \mathbb J_{2n} X S \|_F^2 \\
	&= \| \tilde S - \tilde A \tilde A ^+ \tilde S \|_F^2.
\end{aligned}
\end{equation}
Here $S$ is the matrix of samples $\{ z(t_i) \}_{i=1}^{N}$ in its columns, $\tilde S = X^{1/2} S$, $\tilde A = X^{1/2} A$ and $\tilde A^+ = \mathbb J_{2k}^T \tilde A^T J_{2n}$ is the symplectic inverse of $\tilde A$ with respect to the skew-symmetric matrix $J_{2n} = X^{1/2} \mathbb J_{2n} X^{1/2}$. Note that the symplectic inverse in (\ref{eq:normmor.4}) is a generalization of the symplectic inverse introduced in \Cref{sec:mor.2}. Therefore, we may use the same notation (the superscript $+$) for both. We summarized the properties of this generalization in \Cref{thm:2}. With this notation, the condition (\ref{eq:normmor.2}) turns into $\tilde A ^+ \tilde A = I_{2k}$ which is equivalent to $\tilde A ^T J_{2n} \tilde A = \mathbb J_{2k}$. In other words, this condition implies that $\tilde A$ has to be a $J_{2n}$-symplectic matrix. Finally we can rewrite the minimization (\ref{eq:normmor.3}) as
\begin{equation} \label{eq:normmor.5}
\begin{aligned}
& \underset{\tilde A\in \mathbb{R}^{2n\times 2k}}{\text{minimize}}
& & \| \tilde S - P^\text{symp}_{X,\tilde A}(\tilde S) \|_F, \\
& \text{subject to}
& & \tilde A^T J_{2n} \tilde A = \mathbb J_{2k}.
\end{aligned}
\end{equation}
where $P^\text{symp}_{X,\tilde A} = \tilde A \tilde A^+$ is the symplectic projection with respect to the $X$-norm onto the span of $\tilde A$. At first glance, the minimization (\ref{eq:normmor.5}) might look similar to (\ref{eq:mor.13}). However, since $\tilde A$ is $J_{2n}$-symplectic, and the projection operator depends on $X$, we need to seek an alternative approach to find a near optimal solution to (\ref{eq:normmor.5}). 

As (\ref{eq:mor.13}), direct approaches to solving (\ref{eq:normmor.5}) are impractical. Furthermore, there are no SVD-type methods known to the authors, that solve (\ref{eq:normmor.5}). However, the greedy generation of the symplectic basis can be generalized to generate a near optimal basis $\tilde A$. The generalized greedy method is discussed in \Cref{sec:normmor.2}.

Now suppose that a basis $A=X^{-1/2}\tilde A$, with $\tilde A$ solving (\ref{eq:normmor.5}), is available such that $z \approx Ay$ with $y\in \mathbb R^{2k}$, the expansion coefficients of $z$ in the basis of $A$. Using (\ref{eq:normmor.2}) we may write the reduced system to (\ref{eq:mor.8}) as
\begin{equation} \label{eq:normmor.6}
	\dot y = \mathbb J_{2k}^T A^T X \mathbb J_{2n} X \mathbb{J}_{2n} LAy + \mathbb J_{2k}^T A^T X \mathbb J_{2n} X \mathbb{J}_{2n} \nabla_z f(Ay).
\end{equation}
Since $(\mathbb J_{2k}^T A^T X \mathbb J_{2n} X) A = I_{2k}$, we may use the chain rule to write
\begin{equation} \label{eq:normmor.7}
	\nabla_z H(z) = ( \mathbb J_{2k}^T A^T X \mathbb J_{2n} X )^T \nabla_y H(Ay).
\end{equation}
Finally, as $\nabla_z H(z) = Lz + \nabla_z f(z)$, the reduced system (\ref{eq:normmor.6}) becomes
\begin{equation} \label{eq:normmor.8}
\left\{
\begin{aligned}
	\dot y(t) &= J_{2k} A^T L A y + J_{2k} \nabla_y f(Ay), \\
	y(0) &= \tilde A^+ X^{1/2} z_0,
\end{aligned}
\right.
\end{equation}
where $J_{2k}=\tilde A^+ J_{2n} (\tilde A^+)^T$ is a skew-symmetric matrix. The system (\ref{eq:normmor.8}) is a generalized Hamiltonian system with the Hamiltonian defined as $\tilde H(y) = \frac 1 2 y^TA^TLAy + f(Ay)$. Therefore, a Poisson integrator preserves the symplectic symmetry associated with (\ref{eq:normmor.8}). 


We close this section by summarizing the properties of the symplectic inverse in the following theorem.
\begin{theorem} \label{thm:2}
Let $A\in \mathbb R^{2n\times 2k}$ be a $J_{2n}$-symplectic basis where $J_{2n}\in\mathbb R^{2n}$ is a full rank and skew-symmetric matrix. Furthermore, suppose that $A^{+} = \mathbb{J}_{2k}^T A^T J_{2n}$ is the symplectic inverse. Then the following holds:
\begin{enumerate}
\item $A^+A = I_{2k}$.
\item $(A^+)^T$ is $J_{2n}^{-1}$-symplectic.
\item $\left(\left(\left(A^+\right)^T\right)^+\right)^T = A$.
\item Let $J_{2n}=X^{1/2}\mathbb J_{2n} X^{1/2}$. Then $A$ is ortho-normal with respect to the $X$-norm, if and only if $(A^+)^T$ is ortho-normal with respect to the $X^{-1}$-norm.
\end{enumerate}
\end{theorem}
\begin{proof}
It is straight forward to show all statements using the definition of a symplectic basis.
\end{proof}

\subsection{Greedy generation of a $J_{2n}$-symplectic basis} \label{sec:normmor.2}
In this section we modify the greedy algorithm introduced in \Cref{sec:mor.3} to construct a $J_{2n}$-symplectic basis. Ortho-normalization is an essential step in greedy approaches to basis generation \cite{hesthaven2015certified,quarteroni2015reduced}. Here, we summarize a variation of the GS orthogonalization process, known as the \emph{symplectic GS} process.

Suppose that $\Omega$ is a symplectic form defined on $\mathbb R^{2n}$ such that $\Omega(x,y) = x^T J_{2n} y$, for all $x,y\in \mathbb R^{2n}$ and some full rank and skew-symmetric matrix $J_{2n} = X^{1/2} \mathbb J_{2n} X^{1/2}$. We would like to build a basis of size $2k+2$ in an iterative manner and start with some initial vector, e.g. $e_1 = z_0$. It is known that a symplectic basis is even dimensional \cite{Marsden:2010:IMS:1965128}. We may take $Te_1$, where $T = X^{-1/2} \mathbb J_{2n}^{T}X^{1/2}$, as a candidate for the second basis vector. It is easily checked that $\tilde A_2=[e_1|Te_1]$ is $J_{2n}$-symplectic and consequently, $\tilde A_2$ is the first basis generated by the greedy approach. Next, suppose that $\tilde A_{2k} = [e_1|\dots|e_k|Te_1|\dots|Te_k]$ is generated in the $k$th step of the greedy method and $z\not \in \text{span}\left(\tilde A_{2k}\right)$ is provided. We aim to $J_{2n}$-orthogonalize $z$ with respect to the basis $\tilde A_{2k}$. This means we seek find coefficients $\alpha_i,\beta_i\in \mathbb R$, for $i=1,\dots,k$ such that
\begin{equation} \label{eq:normmor.9}
	\Omega\left( z +\sum_{i=1}^{k} \alpha_i e_i +\sum_{i=1}^{k} \beta_i Te_i, \sum_{i=1}^{k}\bar \alpha_i e_i +\sum_{i=1}^{k} \bar \beta_i Te_i \right) = 0,
\end{equation}
for all possible $\bar \alpha_i,\bar \beta_i \in \mathbb R$, $i=1,\dots,k$. It is easily checked that this problem has the unique solution
\begin{equation} \label{eq:normmor.10}
	\alpha_i = - \Omega(z,Te_i), \quad \beta = \Omega(z,e_i).
\end{equation}
If we take $\tilde z = z -\sum_{i=1}^{k} \Omega(z,Te_i) e_i +\sum_{i=1}^{k} \Omega(z,e_i) Te_i$, then the next candidate pair of basis vectors are $e_{k+1} = \tilde z / \| \tilde z \|_X$ and $Te_{k+1}$. Finally, the basis generated at the $(k+1)$-th step of the greedy method is given by
\begin{equation} \label{eq:normmor.11}
	\tilde A_{2k+2} = [e_1|\dots|e_k|e_{k+1}|Te_1|\dots|Te_k|Te_{k+1}].
\end{equation}
\Cref{thm:3} guarantees that the column vectors of $\tilde A_{2k+2}$ are linearly independent. Furthermore, it is checked easily that $\tilde A_{2k+2}$ is $J_{2n}$-symplectic. We note that the symplectic GS orthogonalization process is chosen due to its simplicity. However, in problems where there is a need for a large basis, this process might be impractical. In such cases, one may use a backward stable routine, e.g. the isotropic Arnoldi method or the isotropic Lanczos method \cite{doi:10.1137/S1064827500366434}.

It is well known that a symplectic basis, in general, is not norm bounded \cite{doi:10.1137/050628519}. The following theorem guarantees that the greedy method for generating a $J_{2n}$-symplectic basis yields a bounded basis.
\begin{theorem} \label{thm:3}
The basis generated by the greedy method for constructing a $J_{2n}$-symplectic basis is ortho-normal with respect to the $X$-norm.
\end{theorem}
\begin{proof}
Let $\tilde A_{2k}=[e_1|\dots,e_k|Te_1|\dots|Te_k]$ be the $J_{2n}$-symplectic basis generated at the $k$th step of the greedy method. Using the fact that $\tilde A_{2k}$ is $J_{2n}$-symplectic, one can check that
\begin{equation} \label{eq:normmor.12}
	[e_i,e_j]_X = [Te_i,Te_j]_X = \Omega(e_i,Te_j)=\delta_{i,j}, \quad i,j=1,\dots,k,	
\end{equation}
and
\begin{equation} \label{eq:normmor.13}
	[e_i,Te_j]_X = \Omega(e_i,e_j) = 0\quad i,j=1,\dots,k,
\end{equation}
where $\delta_{i,j}$ is the Kronecker delta function. This ensures that $\tilde A_{2k}^TX\tilde A_{2k} = I_{2k}$, i.e., $\tilde A_{2k}$ is an ortho-normal basis with respect to the $X$-norm.
\end{proof}
We note that if we take $X=I_{2n}$, then the greedy process generates a $\mathbb J_{2n}$- symplectic basis. With this choice, the greedy method discussed above becomes identical to the greedy process discussed in \Cref{sec:mor.3}, since $T = X^{-1/2}\mathbb J_{2n}^TX^{1/2} = \mathbb J_{2n}^T$.

For identifying the best vectors to be added to a set of basis vectors, we may use similar error functions to those introduced in \Cref{sec:mor.3}. The projection error can be used to identify the snapshot that is worst approximated by a given basis $\tilde A_{2k}$:
\begin{equation} \label{eq:normmor.14}
	z_{k+1} := \underset{z\in\{ z(t_i)\}_{i=1}^{N}}{\text{argmax } }\| X^{1/2}z - P^\text{symp}_{X,\tilde A_{2k}}(X^{1/2}z) \|_2. 
\end{equation}
Alternatively we can use the loss in the Hamiltonian function in (\ref{eq:mor.16}) for parameter dependent problems. We summarize the greedy method for generating a $J_{2n}$-symplectic matrix in \Cref{alg:2}.

\begin{algorithm} 
\caption{The greedy algorithm for generation of a $J_{2n}$-symplectic basis} \label{alg:2}
{\bf Input:} Tolerated projection error $\delta$, initial condition $ z_0$, the snapshots $\mathcal Z = \{X^{1/2} z(t_i)\}_{i=1}^{N}$, full rank matrix $X=X^T>0$
\begin{enumerate}
\item $T \leftarrow X^{-1/2}\mathbb J_{2n}^T X^{1/2}$
\item $z_1 = z(0)$
\item $e_1 \leftarrow z_1/ \| z_1 \|_X$
\item $\tilde A \leftarrow [e_1|Te_1]$
\item $k \leftarrow 1$
\item \textbf{while} $\| z - P^\text{symp}_{X,\tilde A}( z ) \|_2 > \delta$ for all $z \in \mathcal Z$
\item \hspace{0.5cm} $z_{k+1} := \underset{z\in \mathcal Z}{\text{argmax }} \| z - P^\text{symp}_{X,\tilde A}( z ) \|_2$
\item \hspace{0.5cm} $J_{2n}$-orthogonalize $z_{k+1}$ to obtain $e_{k+1}$
\item \hspace{0.5cm} $\tilde A \leftarrow [e_1|\dots |e_{k+1} | Te_1|\dots| Te_{k+1}]$
\item \hspace{0.5cm} $k \leftarrow k+1$
\item \textbf{end while}
\item $A\leftarrow X^{-1/2} \tilde A$
\end{enumerate}
\vspace{0.5cm}
{\bf Output:} $J_{2n}$-symplectic basis $\tilde A$ and the reduced basis $A$
\end{algorithm}

It is shown in \cite{doi:10.1137/17M1111991} that under natural assumptions on the solution manifold of (\ref{eq:mor.8}), the original greedy method for symplectic basis generation converges exponentially fast. We expect the generalized greedy method, equipped with the error function (\ref{eq:normmor.14}), to converge as fast, since the $X$-norm is topologically equivalent to the standard Euclidean norm \cite{friedman1970foundations}, for a full rank matrix $X$.

\subsection{Efficient evaluation of nonlinear terms} \label{sec:normmor.3}
The evaluation of the nonlinear term in (\ref{eq:normmor.8}) still retains a computational complexity proportional to the size of the full order system (\ref{eq:mor.8}). To overcome this, we take an approach similar to \Cref{sec:mor.2}. The DEIM approximation of the nonlinear term in (\ref{eq:normmor.8}) yields
\begin{equation} \label{eq:normmor.15}
	\dot y = J_{2k} A^TLAy + \tilde A ^+ X^{1/2} \mathbb J_{2n} U (\mathcal P^TU)^{-1}\mathcal  P^T \nabla_z f(Ay).
\end{equation}
Here $U$ is a basis constructed from the nonlinear snapshots $\{\nabla_z f(z(t_i))\}_{i=1}^N$, and $\mathcal P$ is the interpolating index matrix \cite{Chaturantabut:2010cz}. As discussed in \Cref{sec:mor.2}, for a general choice of $U$, the reduced system (\ref{eq:normmor.8}) does not retain a Hamiltonian form. Since $(\tilde A^+ X^{1/2}) A = I_{2k}$ applying the chain rule on (\ref{eq:normmor.15}) yields
\begin{equation} \label{eq:normmor.16}
	\dot y = J_{2k} A^TLAy + \tilde A ^+ X^{1/2} \mathbb J_{2n} U (\mathcal P^TU)^{-1} \mathcal P^T (\tilde A^+ X^{1/2})^T \nabla_y f(Ay).
\end{equation}
If we require $U = X^{1/2} (\tilde A^+)^T$ then the complex expression in (\ref{eq:normmor.16}) reduces to
\begin{equation} \label{eq:normmor.17}
	\dot y = J_{2k} A^TLAy + J_{2k} \nabla_y f(Ay),
\end{equation}
and hence we recover the Hamiltonian structure. This yields the reduced system
\begin{equation} \label{eq:normmor.18}
\left\{
\begin{aligned}
	\dot y(t) &= J_{2k} A^TLAy + J_{2k} (\mathcal P^TX^{1/2} (\tilde A^+)^T)^{-1} \mathcal P^T \nabla_z f(z), \\
	y(0) &= \tilde A^+ X^{1/2} z_0.
\end{aligned}
\right.
\end{equation}
We now discuss how to ensure that $X^{1/2} (\tilde A^+)^T$ is a basis for the nonlinear snapshots. Note that if $z \in \text{span}\left(X^{1/2} (\tilde A^+)^T\right)$ then $X^{-1/2} z \in \text{span}\left(( \tilde A^+)^T \right)$. Therefore, it is sufficient to require $(\tilde A^+)^T$ to be a basis for $\{X^{-1/2} \nabla_z f(z(t_i))\}_{i=1}^N$. \Cref{thm:2} suggests that $(\tilde A^+)^T$ is a $J_{2n}^{-1}$-symplectic basis and that the transformation between $\tilde A$ and $(\tilde A^+)^T $ does not affect the symplectic feature of the bases. Consequently, from $A$ we may compute $(\tilde A^+)^T$ and enrich it with snapshots $\{X^{-1/2} \nabla_z f(z(t_i))\}_{i=1}^N$. Once $(\tilde A^+)^T$ represents the nonlinear term with the desired accuracy, we may compute $\tilde A= \left( \left( ( \tilde A^+ )^T \right)^+ \right)^T$ to obtain the reduced basis for (\ref{eq:normmor.18}). Note that \Cref{thm:2} implies that $(\tilde A^+)^T$ is ortho-normal with respect to the $X^{-1}$-norm. This affects the ortho-normalization process. We summarized the process of generating a basis for the nonlinear terms in \Cref{alg:3}.

\begin{algorithm} 
\caption{Generation of a basis for nonlinear terms} \label{alg:3}
{\bf Input:} Tolerated projection error $\delta$, $J_{2n}$-symplectic basis $\tilde A$ of size $2k$, the snapshots $\mathcal G = \{X^{-1/2} \nabla_zf(z(t_i))\}_{i=1}^{N}$, full rank matrix $X=X^T>0$
\begin{enumerate}
\item $T \leftarrow X^{1/2}\mathbb J_{2n}^T X^{-1/2}$
\item compute $(\tilde A^+)^T$
\item \textbf{while} $\| g - P^\text{symp}_{X^{-1},(\tilde A^+)^T}( g ) \|_2 > \delta$ for all $g \in \mathcal G$
\item \hspace{0.5cm} $g_{k+1} := \underset{g\in \mathcal G}{\text{argmax }} \| g - P^\text{symp}_{X^{-1},(\tilde A^+)^T}( g ) \|_2$
\item \hspace{0.5cm} $J_{2n}^{-1}$-orthogonalize $ g_{k+1}$ to obtain $e_{k+1}$
\item \hspace{0.5cm} $(\tilde A^+)^T \leftarrow [e_1|\dots |e_{k+1} | Te_1|\dots| Te_{k+1}]$
\item \hspace{0.5cm} $k \leftarrow k+1$
\item \textbf{end while}
\item $\tilde A \leftarrow \left( \left( ( \tilde A^+ )^T \right)^+ \right)^T$
\end{enumerate}
\vspace{0.5cm}
{\bf Output:} $J_{2n}$-symplectic basis $\tilde A$
\end{algorithm}

\subsection{Offline/online decomposition} \label{sec:normmor.4}
Model order reduction becomes particularly useful for parameter dependent problems in multi-query settings. For the purpose the of most efficient computation, it is important to delineate high dimensional ($\mathcal{O}(n^{\alpha})$) offline computations from low dimensional ($\mathcal{O}(k^{\alpha})$) online ones, for some $\alpha \in \mathbb N$. Time intensive high dimensional quantities are computed only once for a given problem in the offline phase and the cheaper low dimensional computations can be performed in the online phase. This segregation or compartmentalization of quantities, according to their computational cost, is referred to as the offline/online decomposition.

More precisely, one can decompose the computations into the following stages:
\emph{Offline stage:} Quantities in this stage are computed only once and then used in the online stage.
\begin{enumerate}
\item Generate the weighted snapshots $\{ X^{1/2} z(t_i) \}_{i=1}^N$ and the weighted snapshots of the nonlinear term $\{X^{-1/2}. \nabla_zf(z(t_i))\}_{i=1}^N$
%\item Generate weighted snapshots matrix $X^{1/2}S$ and basis $U$ of nonlinear snapshots $\{\nabla_zf(z(t_i))\}_{i=1}^N$ by solving the high dimensional system \cref{eq:mor.8}.
\item Generate a $J_{2n}$-symplectic basis for the solution snapshots and the snapshots of the nonlinear terms, following \Cref{alg:2,alg:3}, respectively.
\item Assemble the reduced order model \cref{eq:mor.12}.
\end{enumerate}
\emph{Online stage:} the reduced model \cref{eq:mor.12} is solved for multiple parameter sets and the output is extracted.

\section{Numerical Results} \label{sec:res}

\section{Conclusion} \label{sec:conc}
In this paper we present a model reduction routine that combines the classic model reduction method with respect to a weighted inner product with the symplectic model reduction. This allows the reduced system to be defined with respect to the norms and inner-products, natural to the problem. Furthermore, the symplectic nature of the reduced system will preserve the Hamiltonian structure of the original system, which result in robustness and enhanced stability in the reduced system.

We demonstrate that including the weighted inner-product in the syplectic model reduction can be viewed as a natural extension of the symplectic method. Therefore, the stability preserving properties of the symplectic method generalize naturally to the new method.

Numerical results suggest the classic model reduction methods with respect to an weighted inner product can help with the boundedness of the system. However, only the symplectic treatment can consistently increase the accuracy of the reduced system. This is consistent with the fact the symplectic methods preserve the Hamiltonian structure.

We also exhibit that to accelerate the evaluation of the nonlinear terms, adopting a symplastic approach is essential. This allows an accurate reduced system that is consistently enhanced when the basis for the nonlinear term is enriched.

Hence, the symplectic model-reduction with respect to a weigther inner product can provide an accurate and robust reduced system that carry the norms and inner products most appropriate to the problem.

\section*{Acknowledgments} We would like to show our sincere appreciation to Dr. Claudia Maria Colciago for the several brainstorming meetings which helped with the development of the main parts of this article. Furthermore, we would like to thank Prof. Bernard Haasdonk who initiated the investigation which led to this article. Finally, we would like to thank Prof. Karen Willcox for hosting Babak Maboudi Afkham at MIT during the composition of this paper. 


\bibliographystyle{siamplain}
\bibliography{references}

\end{document}
