\section{Hamiltonian systems}
\label{sec:hamil}

In this section we discuss the basic concepts of the geometry of symplectic linear vector spaces and introduce Hamiltonian and Generalized Hamiltonian systems.

\subsection{Generalized Hamiltonian systems}
\label{sec:hamil.1}

Let $(\mathbb R^{2n}, \Omega)$ be a symplectic linear vector space, with $\mathbb R^{2n}$ the configuration space and $\Omega:\mathbb R^{2n}\times\mathbb R^{2n} \to \mathbb R$ is a closed, skew-symmetric and non-degenerate 2-form on $\mathbb R^{2n}$. Given a smooth function $H:\mathbb R^{2n} \to \mathbb R$, the so called \emph{Hamiltonian}, the \emph{generalized Hamiltonian system} of evolution reads
\begin{equation} \label{eq:hamil.1}
\left\{
\begin{aligned}
	& \dot z = J_{2n} \nabla_z H(z),  \\
	&  z(0) = z_0.
\end{aligned}
\right.
\end{equation}
Here $z\in \mathbb R^{2n}$ are the configuration coordinates and $J_{2n}$ is a constant, full-rank and skew-symmetric $2n\times 2n$ \emph{structure} matrix such that $\Omega(x,y) = x^TJ_{2n}y$, for all state vectors $x,y\in \mathbb R^{2n}$ \cite{Marsden:2010:IMS:1965128}. Note that there always exists a coordinate transformation $\tilde z = \mathcal T^{-1} z$, with $\mathcal T \in \mathbb R^{2n\times 2n}$, such that $J_{2n}$ takes the form of the \emph{standard} symplectic structure matrix
\begin{equation} \label{eq:hamil.2}
	\mathbb{J}_{2n} = 
	\begin{pmatrix}
	0_n & I_n \\
	-I_n & 0_n
	\end{pmatrix},
\end{equation}
in the new coordinate system \cite{de2006symplectic}.
Here $0_n$ and $I_n$ are the zero matrix and the identity matrix of size $n\times n$, respectively. A central feature of Hamiltonian systems is conservation of the Hamiltonian.

\begin{theorem} \label{thm:1}
\cite{Marsden:2010:IMS:1965128} The Hamiltonian $H$ is a conserved quantity of the Hamiltonian system \eqref{eq:hamil.1} i.e. $H(z(t)) = H(z_0)$ for all $t \geq 0$.
\end{theorem}

Under a general coordinate transformation, the equations of evolution of a Hamiltonian system might not take the form (\ref{eq:hamil.1}). Indeed only transformations which preserve the symplectic form, \emph{symplectic transformations}, preserve the form of a Hamiltonian system \cite{Hairer:1250576}. Suppose that $(\mathbb R^{2n},\Omega)$ and $(\mathbb R^{2k},\Lambda)$ are two symplectic linear vector spaces. A transformation $\mu:\mathbb R^{2n}\to\mathbb R^{2k}$ is a symplectic transformation if
\begin{equation} \label{eq:hamil.3}
	\Omega(x,y) = \Lambda(\mu(x),\mu(y)), \quad \text{for all } x,y\in\mathbb R^{2n}.
\end{equation}
In matrix notation, i.e. when we consider a set of basis vectors for $\mathbb R^{2n}$ and $\mathbb R^{2k}$, a linear symplectic transformation is of the form $\mu(x) = Ax$ with $A\in \mathbb R^{2n\times 2k}$ such that
\begin{equation} \label{eq:hamil.4}
	A^T J_{2n} A = J_{2k}.
\end{equation}
We are interested in a class of symplectic transformations that transform a symplectic structure $J_{2n}$ into the standard symplectic structure $\mathbb J_{2k}$.
\begin{definition} \label{def:symp-mat}
Let $J_{2n}\in \mathbb R^{2n\times 2n}$ be a full-rank skew-symmetric structure matrix. A matrix $A\in\mathbb R^{2n\times 2k}$ is $J_{2n}$-symplectic if
\begin{equation} \label{eq:hamil.5}
A^T J_{2n} A = \mathbb{J}_{2k}.
\end{equation}
\end{definition}
Note that in the literature \cite{Marsden:2010:IMS:1965128,Hairer:1250576}, symplectic transformations refer to $\mathbb{J}_{2n}$-symplectic matrices, in contrast to \Cref{def:symp-mat}.

It is natural to expect a numerical integrator that solves (\ref{eq:hamil.1}) to also satisfy the conservation law expressed in  \Cref{thm:1}. Conventional numerical time integrators, e.g. general Runge-Kutta methods, do not generally preserve the symplectic symmetry of Hamiltonian systems which often result in an unphysical behavior of the solution over long time-integration. \emph{Poisson integrators} \cite{Hairer:1250576} are known to preserve the Hamiltonian of \eqref{eq:hamil.1}. To construct a general Poisson integrator, we seek a coordinate transformation $\mathcal T:\mathbb R^{2n}\to\mathbb R^{2n}$, $\tilde z = T^{-1}z$, such that $J_{2n} = \mathcal T \mathbb J_{2n} \mathcal T^T$. Then, a \emph{symplectic integrator} can preserve the symplectic structure of the transformed system. The \emph{St\"ormer-Verlet} scheme is an example of a second order symplectic time-integrator given as
\begin{equation} \label{eq:hamil.6}
	\begin{aligned}
	q_{m+1/2} &= q_m + \frac{\Delta t} 2 \cdot \nabla_p \tilde H(p_m,q_{m+1/2}), \\
	p_{m+1} &= p_m - \frac{\Delta t} 2  \cdot \left( \nabla_q \tilde H(p_m,q_{m+1/2}) + \nabla_{q} \tilde H(p_{m+1},q_{m+1/2}) \right), \\
	q_{m+1} &= q_{m+1/2} + \frac{\Delta t} 2  \cdot  \nabla_p \tilde H(p_{m+1},q_{m+1/2}).
	\end{aligned}
\end{equation}
Here, $\tilde z = (q^T,p^T)^T$, $\tilde H(\tilde z) = H(T^{-1}z)$, $\Delta t$ denotes a uniform time step-size, and $q_m \approx q(m\Delta t)$ and $p_m \approx p(m\Delta t)$, $m \in \mathbb{N} \cup \{ 0\}$, are approximate numerical solutions. Note that it is important to use a backward stable method to compute the transformation $\mathcal T$. In this paper we use the symplectic Gaussian elimination method with complete pivoting to compute the decomposition $J_{2n} = \mathcal T \mathbb J_{2n} \mathcal T^T$. However, one may use a more computationally efficient method, e.g., a Cholesky-like factorization proposed in \cite{benner:chol} or the isotropic Arnoldi/Lanczos methods \cite{doi:10.1137/S1064827500366434}. There are a few known numerical integrators that preserve the symplectic symmetry of a generalized Hamiltonian system without requiring the computation of the transformation matrix $\mathcal T$ \cite{Hairer:1250576}. The implicit midpoint rule
\begin{equation} \label{eq:hamil.7}
	z_{m+1} = z_{m} + \Delta t \cdot J_{2n} \nabla_z H \left( \frac{z_{m+1} + z_m}{2} \right),
\end{equation}
for \cref{eq:hamil.1} is an example of such integrators. For more on the construction and the applications of Poisson/symplectic integrators, we refer the reader to \cite{Hairer:1250576,bhatt2017structure}.
