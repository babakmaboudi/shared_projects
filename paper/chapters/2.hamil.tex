\section{Hamiltonian systems}
\label{sec:hamil}

Let $(\mathcal Z, \Omega)$ be a symplectic linear vector space \cite{Marsden:2010:IMS:1965128}, with $\mathcal Z \subset \mathbb R^{2n}$ the configuration space and $\Omega:\mathbb R^{2n}\times\mathbb R^{2n} \to \mathbb R$ a closed, skew-symmetric and non-degenerate 2-form on $\mathcal Z$. Given a smooth function $H:\mathbb R^{2n} \to \mathbb R$, the so called \emph{Hamiltonian}, the \emph{Hamiltonian system} of evolution reads
\begin{equation} \label{eq:hamil.1}
\left\{
\begin{aligned}
	& \dot z = \mathbb J_{2n} \nabla_z H(z),  \\
	&  z(0) = z_0.
\end{aligned}
\right.
\end{equation}
Here $z\in \mathcal Z$ \edit{is the state vector} and $\mathbb J_{2n}$ is the \emph{symplectic} or \emph{canonical} matrix
\begin{equation} \label{eq:hamil.2}
	\mathbb{J}_{2n} = 
	\begin{pmatrix}
	0_n & I_n \\
	-I_n & 0_n
	\end{pmatrix},
\end{equation}
such that $\Omega(x,y) = x^T\mathbb J_{2n}y$, for all state vectors $x,y\in \mathbb R^{2n}$ \cite{Marsden:2010:IMS:1965128}. Here $0_n$ and $I_n$ are the zero matrix and the identity matrix of size $n\times n$, respectively.

A general coordinate transformation does not in general preserve canonical properties of a Hamiltonian system (\ref{eq:hamil.1}). Indeed only transformations which preserve the symplectic form, \emph{symplectic transformations}, preserve the form of a Hamiltonian system \cite{Hairer:1250576}.

Suppose that $(\mathcal Z,\Omega)$ and $(\mathcal Y,\Lambda)$, with $\mathcal Z \subset \mathbb R^{2n}$ and $\mathcal Y \subset \mathbb R^{2n}$, are two symplectic linear vector spaces with a canonical basis. A transformation $\mu:\mathcal Z\to\mathcal Y$ is symplectic if
\begin{equation} \label{eq:hamil.3}
	\Omega(x,y) = \Lambda(\mu_z(z)x,\mu_z(z)y), \quad \text{for all } x,y\in\mathcal Z,
\end{equation}
where subscript $z$ denotes the gradient. Therefore symplectic 2-forms of a pair of vectors and their images under a symplectic transformation are equal. It is straightforward to see that a linear transformation $\mu(x) = Ax$, with $A\in \mathbb R^{2n\times 2k}$ and $\Omega = \Lambda$, is symplectic if
\begin{equation} \label{eq:hamil.4}
	A^T \mathbb J_{2n} A = \mathbb J_{2k}.
\end{equation}

%We are interested in a class of symplectic transformations that transform a symplectic structure $J_{2n}$ into the standard symplectic structure $\mathbb J_{2k}$.
%\begin{definition} \label{def:symp-mat}
%Let $J_{2n}\in \mathbb R^{2n\times 2n}$ be a full-rank skew-symmetric structure matrix. A matrix $A\in\mathbb R^{2n\times 2k}$ is $J_{2n}$-symplectic if
%\begin{equation} \label{eq:hamil.5}
%A^T J_{2n} A = \mathbb{J}_{2k}.
%\end{equation}
%\end{definition}
%Note that in the literature \cite{Marsden:2010:IMS:1965128,Hairer:1250576}, symplectic transformations refer to $\mathbb{J}_{2n}$-symplectic matrices, in contrast to \Cref{def:symp-mat}.

A central feature of Hamiltonian systems is preservation of the Hamiltonian and the symplectic form by the flow of the system.
\begin{theorem} \label{thm:1}
\cite{Marsden:2010:IMS:1965128} The Hamiltonian $H$ is a conserved quantity of the Hamiltonian system \eqref{eq:hamil.1} i.e. $H(z(t)) = H(z_0)$ for all $t \geq 0$. Moreover, the flow $\phi_{t,H}:z_0 \to z(t;z_0)$ of a Hamiltonian system is symplectic, i.e.,
$$\partial_z \phi_{t,H}(z)^T \mathbb J_{2n} \partial_z \phi_{t,H}(z) = \mathbb J_{2n}.$$
\end{theorem}

Preservation of phase space area or volume is a consequence of both, a Hamiltonian system with the standard symplectic structure and a symplectic transformation. Therefore, it is natural to expect a numerical integrator that solves (\ref{eq:hamil.1}) to also satisfy the conservation laws expressed in  \Cref{thm:1}. Conventional numerical time integrators, e.g. general Runge-Kutta methods, do not generally preserve the symplectic symmetry of Hamiltonian systems which often result in an unphysical behavior of the solution over long time-integration. The \emph{St\"ormer-Verlet} scheme is an example of a second order symplectic time-integrator given as
\editA{
\begin{equation} \label{eq:hamil.6}
	\begin{aligned}
	q_{m+1/2} &= q_m + \frac{\Delta t} 2 ~ \nabla_p  H(q_{m+1/2},p_m), \\
	p_{m+1} &= p_m - \frac{\Delta t} 2  ~ \left( \nabla_q  H(q_{m+1/2},p_m) + \nabla_{q}  H(q_{m+1/2},p_{m+1}) \right), \\
	q_{m+1} &= q_{m+1/2} + \frac{\Delta t} 2  ~  \nabla_p  H(q_{m+1/2},p_{m+1}).
	\end{aligned}
\end{equation}
Here, $ z = (q^T,p^T)^T$, $ H(z) = H(q,p)$, $\Delta t$} denotes a uniform time step-size, and $q_m \approx q(m\Delta t)$ and $p_m \approx p(m\Delta t)$, $m \in \mathbb{N} \cup \{ 0\}$, are approximate numerical solutions.

Note that \Cref{thm:1} is also valid for a Hamiltonian system in a transformed coordinate system, associated with a skew-symmetric and full rank structure matrix $J_{2n}$. Such Hamiltonian systems also carry symmetries, e.g., the symmetry expressed in \Cref{thm:1} or the preservation of the phase space volume \cite{Hairer:1250576}. However, to ensure a robust and long time-integration, geometric numerical integration of Hamiltonian systems that exploits such symmetries is preferred and better established in a canonical coordinate systems \cite{Hairer:1250576,bhatt2017structure}.  For more on the construction and the applications of symplectic and geometric numerical integrators, we refer the reader to \cite{Hairer:1250576,bhatt2017structure}.
