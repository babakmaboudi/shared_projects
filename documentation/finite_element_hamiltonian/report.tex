\title{Model Reduction of Finite Element Hamiltonian Systems With Respect to The Energy Norm}
\author{
        Babak Maboudi Afkham \\
                Department of Mathematics\\
        EPFL\\ \\
	Ashish Bhatt \\
		University of Stuttgart
}
\date{\today}

\documentclass[12pt]{article}

\usepackage{amsmath}
\usepackage{amssymb}
\usepackage{todonotes}
\usepackage{cleveref}
\usepackage{algorithm}

\begin{document}
\maketitle

\begin{abstract}
Here we summarize the basic concepts on how we generalize the model reduction with respect to energy norm to Hamiltonian systems.
\end{abstract}

\section{Preliminaries}

\subsection{Finite Element Formulation}

Consider the wave equation:
\begin{equation} \label{eq:1}
\begin{aligned}
	\partial_t q - p &= 0, \\
	\partial_t p - \Delta q &= f,\\
	q(x,0) &= q0(x),\\
	p(x,0) &= p0(x).
\end{aligned}
\end{equation}
defined on the domain $\Omega\times [0,T] \subset \mathbb R^n \times \mathbb R$ We assume that the solution $(q,p)$ to the system of differential equation (\ref{eq:1}) belongs to $H^1_{\text{per}}\times H^1_{\text{per}}$ where
\begin{equation} \label{eq:4}
	H^1_{\text{per}} = \{ u \in L^2 : \|\nabla u\| \in L^2 \text{ and } u \text{ is periodic on }\Omega \}.
\end{equation}
We denote $(\cdot,\cdot)$ to be $L^2$ inner product. The solution to (\ref{eq:1}) also satisfies the weak form of finding $(q,p)$ such that

\begin{equation} \label{eq:3}
\begin{aligned}
	(\partial_t q , u) - (p,u) &= 0, \\
	(\partial_t p , v) + (\nabla q, \nabla v) &= 0.
\end{aligned}
\end{equation}
The semi-discrete mixed formulation of (\ref{eq:4}) is to find $(q_h,p_h):[0,T]\times[0,T]\to U_h\times V_h$ such that
\begin{equation} \label{eq:4}
\begin{aligned}
	(\partial_t q_h , u_h) - (p_h,u_h) &= 0, \\
	(\partial_t p_h , v_h) + (\nabla q_h, \nabla v_h) &= 0.
\end{aligned}
\end{equation}
where $U_h$ and $V_h$ are finite dimensional linear subspaces of $H^1_{\text{per}}$. Let $\{\phi_i\}_1^{\text{dim}(U_h)}$ and $\{\psi_i\}_1^{\text{dim}(V_h)}$ be the basis functions for $U_h$ and $V_h$ respectively. We define the mass matrices
\begin{equation} \label{eq:5}
\begin{aligned}
	M_{i,j}^q &= (\phi_j,\phi_i), \\
	M_{i,j}^p &= (\psi_j,\psi_i).
\end{aligned}
\end{equation}
Further we define the stiffness matrices
\begin{equation} \label{eq:6}
\begin{aligned}
	K^q_{i,j} &= (\phi_j,\psi_i), \\
	K^p_{i,j} &= (\nabla \psi_j,\nabla \phi_i).
\end{aligned}
\end{equation}
The semi-discrete form (\ref{eq:4}) also satisfies the system of ordinary differential equations
\begin{equation} \label{eq:7}
\begin{aligned}
	M^q q_t - K^q p &= 0, \\
	M^p p_t + K^p q &= M^p f.
\end{aligned}
\end{equation}

The energy corresponding to the Hamiltonian system (\ref{eq:1}) is defined by
\begin{equation} \label{eq:8}
	H(q,p) = \frac 1 2 (p,p) + \frac 1 2 (\nabla q, \nabla q).
\end{equation}
The Hamiltonian defines an inner product on $(\cdot,\cdot)_H : H^1_{\text{per} }\times H^1_{\text{per} }\to \mathbb R$ denoted by
\begin{equation} \label{eq:9}
	( (q_1,p_1) , (q_2,p_2) )_H = \frac 1 2 (p_1,p_2) + \frac 1 2 (\nabla q_1, \nabla q_2),
\end{equation}
and the corresponding energy norm $\| (q,p) \|_H = \sqrt{ H(q,p) }$.

An essential feature of Hamiltonian systems is the conservation of the energy and how it evolves under numerical time-integration. For a solution $(q,p)$ to the Hamiltonian equation (\ref{eq:1}) we have:
\begin{equation} \label{eq:10}
\begin{aligned}
	\frac{d}{dt} \|(q,p) \|_H^2 &= \frac 1 2 \left( \frac{d}{dt}(p,p) + \frac{d}{dt}(\nabla q,\nabla q) \right) \\
	& = (\partial_t p, p) + (\nabla \partial_t q , \nabla q) \\
	& = ( \Delta q + f , p ) + (\nabla q , \nabla q) \\
	& = (f,p) - (\nabla q , \nabla q) + (\nabla q , \nabla q) \\
	& = (f,p)
\end{aligned}
\end{equation}
By taking the integral over $[0,T]$ we obtain
\begin{equation} \label{eq:11}
	H(T) = H(0) + \int_0^T (f,p)\ dt.
\end{equation}
Now applying the Cauchy-Schwartz inequality yields,
\begin{equation} \label{eq:12}
\begin{aligned}
	H(T) &\leq H(0) + \int_0^T \| f(\cdot,t) \|_{L^2} \cdot \| p(\cdot,t) \|_{L^2} \ dt \\
	&\leq H(0) + \underset{0\leq t < T}{\sup(H)} \cdot \int_0^T \| f(\cdot,t) \|_{L^2} \ dt
\end{aligned}
\end{equation}
where the last inequality is due to the fact that the energy inner product defines a norm on $H^1_{\text{per}}$.

\subsection{Circumventing non-Hamiltonianness of \cref{eq:7}}
Since the equation is not-Hamiltonian unless $M^{-1}K$ commute, we propose the following semi-discrete formulation of \eqref{eq:1}:
\begin{equation}
\begin{aligned}
	(\partial_{tt} \psi_h , u_h) + (\nabla \psi_h, \nabla u_h) = (\mathfrak{f},u_h).
\end{aligned}
\end{equation}
where $\psi_h,\mathfrak{f}:[0,T] \to U_h \subset H^1_{per}$. Let $\{\phi_i\}_1^{dim(U_h)}$ be the basis of $U_h$ and define the mass and stiffness matrices to be
\begin{align}
M_{i,j} &= (\phi_j,\phi_i),\\
K_{i,j} &= (\nabla \phi_j,\nabla \phi_i).
\end{align}
Then the following time-continuous equation follows
\begin{align}
M\ddot q +K q = M f
\end{align}
Here $q$ and $f$ are suitable finite coefficient functions in the basis expansion. This is a Canonical Hamiltonian system
\begin{align}
\dot x = J (Lx + h)
\end{align}
with 
$$x = \begin{bmatrix}
q \\ p
\end{bmatrix},$$ $J$ the canonical structure matrix,
$$L = \begin{bmatrix}
K & \\ & M^{-1}
\end{bmatrix},$$
 $$h = \begin{bmatrix}
-Mf \\ 0
\end{bmatrix},$$
and the Hamiltonian
$$H = \frac{1}{2}(\nabla \psi, \nabla \psi) + \frac{1}{2}(\psi_t,\psi_t) +(\psi,\mathfrak{f})= \frac{1}{2}x^TLx+x^Th.$$

Further study:
\begin{enumerate}
\item $L$ seems to be the natural weight matrix $X$ for energy norm.
\end{enumerate}

\subsection{Energy Preservation in Semi-Discrete mixed Formulation}
Let $U_h\in U$ and $V_h\in V$ be finite dimensional proper subspaces of $U$ and $V$ respectively. Furthermore, suppose that $\pi_u:U\to U_h$ and $p_v:V\to V_h$ be the $L^2$ projection operators. By adding and subtracting $(\pi_Uq,u_h)$ and $(\pi_Vp,v_h)$ to the semi-discrete mixed formulation (\ref{eq:4}) we obtain
\begin{equation} \label{eq:13}
\begin{aligned}
	(\dot q,u_h) + (\pi_U\dot q,u_h) - (\pi_U \dot q, u_h) - (p,u_h) + (\pi_V p , u_h) - (\pi_V p , u_h) &= 0 \\
	(\dot p,v_h) + (\pi_V \dot p , v_h) - (\pi_V \dot p , vh) + (\nabla q , \nabla v_h) + (\nabla \pi_U q , \nabla v_h ) - (\nabla \pi_U q , \nabla v_h ) &= (f,v_h)
\end{aligned}
\end{equation}

Having in mind that $q - \pi_U q$ and $p - \pi_V p$ are orthogonal to $U_h$ and $V_h$, respectively, we can omit many terms from above to obtain
\begin{equation} \label{eq:14}
\begin{aligned}
	(\pi_U\dot q,u_h) - (\pi_V p , u_h) &= (\pi_U \dot q - \dot q , u_h) \\
	(\pi_V \dot p, v_h) + (\nabla \pi_U q, \nabla v_h) &= (\pi_V \dot p - \dot p, v_h) + (f,v_h)
\end{aligned}
\end{equation}
Now if we add the original weak form (\ref{eq:4}) to the above we retrieve
\begin{equation} \label{eq:15}
\begin{aligned}
	(\pi_U \dot q - \dot q_h , u_h) - (\pi_V p - p_h, u_h ) &= (\pi_U \dot q - \dot q , u_h) \\
	(\pi_V \dot p - \dot p_h, v_h) + (\nabla \pi_U q - \nabla q_h , \nabla v_h) &= (\pi_V \dot p - \dot p, v_h)
\end{aligned}
\end{equation}
We define new variables $\theta = \pi_U q - q_h \in U_h$, $\rho = \pi_V p - p_h\in V_h$, $\mu = \pi_U q - q$ and  $\xi = \pi_V p  - p$. Then equation (\ref{eq:15}) turns into
\begin{equation} \label{eq:16}
\begin{aligned}
	(\dot \theta , u_h) - (\rho, u_h ) &= (\dot \mu , u_h) \\
	(\dot \rho,v_h) + (\nabla \theta , \nabla v_h) &= (\dot \xi, v_h)
\end{aligned}
\end{equation}

To bound this we use the energy norm
\begin{equation} \label{eq:17}
\begin{aligned}
	\frac d {dt} H(\theta , \rho)^2 = \frac d {dt} \| (\theta,\rho) \|^2 &= (\dot \rho,\rho) + (\nabla \dot \theta,\nabla \theta) \\
	&= (\dot \xi,\rho) - (\nabla \theta , \nabla \rho) - (\rho , \Delta \theta) - (\dot \mu , \Delta \theta) \\
	&= (\dot \xi, \rho) + (\nabla \dot \mu , \nabla \theta) \\
	&= (\dot \xi, \rho) + (\dot \mu, \theta)_{\nabla}
\end{aligned}
\end{equation}
where $(\cdot,\cdot)_{\nabla} = (\nabla \cdot,\nabla \cdot)$. Finally by applying the Cauchy inequality we obtain
\begin{equation} \label{eq:18}
\begin{aligned}
	H^2(\theta,\rho)^2(T) &\leq \int_0^T \| \dot \xi \| \| \rho \| + \| \dot \mu\|_{\nabla} \|\theta \|_\nabla \ dt \\
	& \leq \sqrt{2} \int_0^T \underbrace{( \| \rho \| + \|\theta \|_\nabla )}_{H(\theta,\rho)} ( \| \dot \xi \| + \| \dot \mu\|_{\nabla} ) \ dt \\
	& \leq \sqrt{2}\cdot \underset{0\leq t < T}{\sup H(\theta,\rho)} \int_0^T \| \dot \xi \| + \| \dot \mu\|_{\nabla}  \ dt
\end{aligned}
\end{equation}
Here we used the fact that $H(\theta,\rho)(0) = 0 $. And now by a theorem in [Symplectic-mixed finite element approximation of linear acoustic wave equations, Robert C. Kirby · Thinh Tri Kieu] we get
\begin{equation} \label{eq:19}
	H^2(\theta,\rho)(T) \leq \sqrt 2 \int_0^T \| \dot \xi \| + \| \dot \mu\|_{\nabla}  \ dt
\end{equation}
By assuming regularity on $\dot \xi$ and $\dot \mu$ and with theory of interpolation we can achieve several error estimations on the Finite Element solution.

\subsection{Error Estimation in the Energy Norm}
We define the error in the energy/Hamiltonian norm as
\begin{equation} \label{eq:20}
	\epsilon(t) = \| (q-q_h,p-p_h) \|_H
\end{equation}
Using the triangle inequality, we obtain
\begin{equation} \label{eq:21}
	\sup \epsilon(t) \leq \sup \underbrace{ \| (\mu, \xi) \|_H }_{I} + \sup \underbrace{ \| (\theta, \rho) \|_H }_{II}
\end{equation}
For part $I$ we can directly apply error estimation in the theory of interpolation, under regularity assumptions on $q$ and $p$. Also for part $II$ we can apply the error estimate obtain in (\ref{eq:19}).

\section{Model Reduction With Respect to the Energy Inner Product}

The energy norm appears naturally in the error analysis of the finite element methods. Suppose that $a(\cdot,\cdot)$ is the bilinear form corresponding to the variational formulation
\begin{equation} \label{eq:22}
	a(u,v) = L(v), \quad u,v \in V
\end{equation}
Where $V$ is some appropriate Hilbert space. The finite element discretization of equation (\ref{eq:22}) is
\begin{equation} \label{eq:23}
	a(u_h,v_v) = L(v_h), \quad u_h,v_h\in V_h \subset V
\end{equation}
The energy inner product associated to (\ref{eq:23}) is defined as
\begin{equation} \label{eq:24}
	(u_h,v_h)_a = a(u_h,v_h),
\end{equation}
which implies the energy norm $\| \cdot \|_X$. In vector notation we have
\begin{equation} \label{eq:25}
	(u_h,v_h)_a = \bar u^T X \bar v,
\end{equation} 
where $\bar u$ and $\bar v$ are expansion coefficients of $u_h$ and $v_h$ in the finite element basis, and $X$ is a positive definite matrix, usually taken to be the stiffness matrix. Note that we can rewrite an energy norm in terms of the 2-norm as $\| \bar u \|_X = \| X^{1/2} u \|_2$.

The energy inner product induces a projection. Energy projection of function $u_h$ onto $e$ reads
\begin{equation} \label{eq:26}
	(u_h,e)_a \cdot e = \bar u_h^T X \bar e \cdot \bar e = \bar e \bar e^T X u_h. 
\end{equation}
Therefore the matrix $\bar e\bar e^T X$ is the energy projection operator in the matrix notation. Now suppose that $W = [ w_1,\dots,w_k ]$ is an expansion coefficients of a basis with respect to some finite element basis. Then the energy projection of a function $s$ onto the span space of $W$ would be $WW^TXs$. It is often desirable to find a basis $W$ that minimizes the energy projection error of a set of functions $\{s_1,\dots,s_N\}$. We have
\begin{equation} \label{eq:27}
\begin{aligned}
	\min \sum_{i=1}^N \| s_i - WW^TXs_i \|_X &= \min \sum_{i=1}^N \| X^{1/2} s_i - X^{1/2 }WW^TXs_i \|_2 \\
	&= \min \sum_{i=1}^N \| \tilde s_i - \tilde W \tilde W^T\tilde s_i \|_2 \\
	&= \min \| \tilde S - \tilde W \tilde W^T \tilde S \|_2.
\end{aligned}
\end{equation}
Where $\tilde W = X^{1/2} W$ and $\tilde s_i = X^{1/2} s_i$ and $\tilde S$ is the matrix containing $\tilde s_i$. The Smidth-Mirskey theorem implies that the solution to the above minimization is the truncated singular value decomposition of the matrix $\tilde S$.

\subsection{Symplectic Model Reduction With Respect to an Energy Inner Product}
To fit the energy norm into the symplectic framework, we need to modify the energy projection operator. Suppose that $A$ contains a set of basis vectors (the expansion coefficients of a set of functions in a FEM basis) in its column space. For now we assume this basis is even dimensional. We define a sympelctic projection onto the span of $A$ with respect to the energy weight $X$ as
\begin{equation} \label{eq:28}
	P(s) = AA^\times Xs,
\end{equation}
Where $J$ is the standard symplectic matrix and $A^\times$ is defined as
\begin{equation} \label{eq:29}
	A^\times = J^T A^T X J.
\end{equation}
Note that if we ensure
\begin{equation} \label{eq:30}
	A^\times XA=I,
\end{equation}
with $I$ the identity matrix, then we see that the operator $AJ^TA^TXJX$ becomes idempotent since
\begin{equation} \label{eq:31}
\begin{aligned}
	(AJ^TA^TXJX)^2 &= A\underbrace{J^TA^TXJ}_{A^\times}XAJ^TA^TXJX \\
	&= A\underbrace{A^\times XA}_{I}J^TA^TXJX\\
       	&= AJ^TA^TXJX.
\end{aligned}
\end{equation}
This means that $AJ^TA^TXJX$ is a projection operator onto the span space of $A$. Now if we require the energy norm of the projection of a snapshot matrix $S$ onto the span of $A$ to be minimized with respect to an energy norm we would have
\begin{equation} \label{eq:32}
	\min \| S - AJ^TA^TXJX S \|_X = \min \| X^{1/2} S - X^{1/2} AJ^TA^TXJX S \|_2.
\end{equation}
We define $\tilde S = X^{1/2}S$, $\tilde A = X^{1/2} A$ and the skew-symmetric matrix $\tilde J = X^{1/2} J X^{1/2}$. Then equation (\ref{eq:31}) turns into
\begin{equation} \label{eq:33}
	\min \| \tilde S - \tilde A J^T {\tilde A}^T \tilde J \tilde S\|_2.
\end{equation}
Finally if we define the pseudo inverse ${\tilde A }^+ = J^T A^T X^{1/2} \tilde J = J^T {\tilde A}^T \tilde J$ then the minimization (\ref{eq:31}) is equivalent to
\begin{equation} \label{eq:34}
	\min \| \tilde S - \tilde A {\tilde A}^+ \tilde S \|_2.
\end{equation}
Note that condition (\ref{eq:30}) is equivalent to
\begin{equation} \label{eq:35}
	{\tilde A}^+ \tilde A = I,
\end{equation}
which is satisfied when
\begin{equation} \label{eq:36}
	\tilde A ^T \tilde J \tilde A = J.
\end{equation}
\todo[inline]{did you mean $\tilde A ^T \tilde J \tilde A = \tilde J$? If yes then, this will mean that $\tilde A ^+ \tilde J (\tilde A ^+)^T = \tilde J$, which is still not canonical but closer.}
The later condition holds when $\tilde A$ is a Poisson transformation. Therefore the minimization (\ref{eq:32}) is now rewritten as
\begin{equation} \label{eq:symplectic-basis}
\begin{aligned}
	& \min \| \tilde S - \tilde A {\tilde A}^+ \tilde S \|_2, \\
	&\text{subject to } \tilde A ^T \tilde J \tilde A = J.
\end{aligned}
\end{equation}

\subsection{Model Reduction with a Symplectic and Energy Projected Basis}
Suppose that the FEM discretization of a linear Hamiltonian system takes the form
\begin{equation}
	\dot x = J L x,
\end{equation}
where $x$ is the expansion coefficients of the FEM basis functions and $L$ is some linear positive definite square 
\todo[prepend]{$L$ should also be symmetric for the system to be Hamiltonian.} matrix. Let $A$ be the basis to a reduced subspace such that $x \approx Ay$ where $y$ is the expansion coefficients of $x$ in the basis of $A$. This implies
\begin{equation}
	A \dot y = J L A y.
\end{equation}
Multiplying both sides with $A^\times X$ yields
\begin{equation}
	\dot y = A^\times X J L A y,
\end{equation}
due to the condition (\ref{eq:30}). Having in mind that $L x = \nabla_x H(x)$ for some Hamiltonian function $H$, we recover
\begin{equation}
	\nabla_x H(x) = \nabla_x H(Ay) = ( A^\times X )^T \nabla_y H(Ay).
\end{equation}
This implies that
\begin{equation}
	\dot y = A^\times X J (A^\times X)^T A^T L A y = A^\times X J X (A^\times)^T A^T L A y,
\end{equation}
which can be simplified to the system
\begin{equation} \label{eq:reduced-poisson}
	\dot y = \tilde A ^+ \tilde J (\tilde A ^+)^T  A^T L A y.
\end{equation}
This is a Poisson system since $\tilde A ^+ \tilde J (\tilde A ^+)^T$ is skew-symmetric. A Poisson integrator can therefore preserve the Hamiltonian along integral curves.

\subsection{Greedy Generation of a Generalized Symplectic Basis}
A proper orthogonalization method is at the core of the generation of a reduced basis. We define by $[\cdot,\cdot]$ and $[\cdot,\cdot]_{X}$ a standard Euclidean inner production and the weighted inner product on a Euclidean space, respectively. Further we define
\begin{equation}
	<v_1,v_2>_{J} = v_1^T J v_2,\quad v_1,v_2 \in \mathbb R^{2k}, 
\end{equation}
to be the symplectic form with respect to the skew-symmetric matrix $J$. Suppose that $\tilde A_{2k}=[e_1,\dots,e_k,f_1,\dots,f_k]$ is a given generalized symplectic basis in the sense of (\ref{eq:36}) with respect to the skew-symmetric matrix $\tilde J$ and the weight matrix $X$. Furthermore, assume $z\not \in \text{span} (\tilde A_{2k})$ is given. We wish to $\tilde J$-orthogonalize $z$ with respect to $\tilde A_{2k}$ and seek $\alpha_i,\beta_i$, $i=1,\dots,k$ such that
\begin{equation}
	< z + \sum \alpha_i e_i + \sum \beta_i f_i , \sum \bar \alpha_i e_i + \sum \bar \beta_i f_i >_{\tilde J},
\end{equation}
for all possible $\bar \alpha_i , \bar \beta_i$. It can be easily checked that
\begin{equation}
	\alpha_i = -<z,f_i>_{\tilde J}, \quad \beta_i = <z,e_i>_{\tilde J}.
\end{equation}
Now we define the modified vector
\begin{equation}
	\tilde z = z - \sum <z,f_i>_{\tilde J} e_i + \sum <z,e_i>_{\tilde J} f_i.
\end{equation}
Finally, we take $e_{k+1} = \tilde z / \| \tilde z \|_X$. Note that we have a freedom in taking $f_{k+1}$. If we take
\begin{equation}
	f_{k+1} = \tilde J ^{\times} e_{k+1},
\end{equation}
where $\tilde J ^{\times} = X^{-1/2} J_{n}^T X^{1/2}$, then it can be seen that $\tilde A_{2(k+1)}$ is a generalized symplectic matrix in the sense of (\ref{eq:36}). Also if we take $f_i = \tilde J ^{\times} e_i$, $i=1,\dots,k$, then $A_{2(k+1)}$ is an orthonormal basis with respect to the $X$ norm: we have
\begin{equation}
		[e_i,e_j]_X = [\tilde J ^{\times}e_i,\tilde J ^{\times}e_j]_X = <e_i,\tilde J ^{\times} e_j>_{\tilde J} = \delta_{i,j}, \quad i=1,\dots,k,
\end{equation}
and
\begin{equation}	
		[e_i,\tilde J ^{\times} e_i]_X = <e_i,e_j>_{\tilde J} = 0, \quad i=1,\dots,k.
\end{equation}
We summarize the symplectic greedy approach in the following Algorithm \ref{alg:1}.

\begin{algorithm} 
\caption{The greedy algorithm for generation of a generalized symplectic basis} \label{alg:1}
{\bf Input:} Tolerated projection error $\delta$, initial condition $ \tilde z_0 = X^{1/2} z_0$
\begin{enumerate}
\item $t^1 \leftarrow t=0$
\item $e_1 \leftarrow \tilde z_0$
\item $A \leftarrow [e_1,\tilde J^{\times}_{2n}e_1]$
\item $k \leftarrow 1$
\item \textbf{while} $\| \tilde z(t) - \tilde A{\tilde A}^+\tilde z(t) \|_2 > \delta$ for all $t \in [0,T]$
\item \hspace{0.5cm} $t^{k+1} := \underset{t\in [0,T]}{\text{argmax }} \| \tilde z(t) - \tilde A{\tilde A}^+\tilde z(t) \|_2$
\item \hspace{0.5cm} $\tilde J$-orthogonalize $ \tilde z(t^{k+1})$ to obtain $e_{k+1}$
\item \hspace{0.5cm} $A \leftarrow [e_1,\dots ,e_{k+1} , \tilde J^{\times}e_1,\dots,\tilde J^{\times}e_{k+1}]$
\item \hspace{0.5cm} $k \leftarrow k+1$
\item \textbf{end while}
\end{enumerate}
\vspace{0.5cm}
{\bf Output:} Symplectic basis $A$.
\end{algorithm}

Note that for parametric Hamiltonian system we may use the error in the Hamiltonian as a cheap surrogate to the projection error to accelerate the basis selection procedure. It is worth to mention that if take $X = I$, the identity matrix, then the entire process collapses to the original symplectic model reduction and the original symplectic greedy method.

\subsection{Model Reduction of Nonlinear terms (SDEIM)}

An essential part of model reduction of non-linear terms is the availability of the chain rule on the weak formulation. Suppose that a discretized Hamiltonian system takes the form.
\begin{equation} \label{eq:2.4.1}
	\dot x = J (L x + g(x) ) 
\end{equation}
for some $g(x) = \nabla_x H(x)$. The discrete empirical interpolation method (DEIM) for approximating the nonlinear function yields
\begin{equation} \label{eq:2.4.2}
	\dot y = A^{\times} X J LAy + A^{\times} X J V(P^TV)^{-1} P^T g(Ay).
\end{equation}
Where $V$ contains the basis to the nonlinear snapshots of $g$. The linear term in the equation above can be treated following waht is presented in the previous subsection. Note that for (\ref{eq:2.4.2}) to be a Poisson system we require the nonlinear term to be of the form $\tilde A ^+ \tilde J (\tilde A ^+)^T \nabla_y G(y)$, for some scalar function $G$. This can be obtained by taking $V = X (A^{\times})^T$ because
\begin{equation}
 	g(Ay) = \nabla_x H(Ay) = X (A^{\times})^T \nabla_y H(Ay).
\end{equation}
The question that remains, is how can we require $V = X (A^{\times})^T$. To answer we need to state the following lemma.

\textbf{Lemma. }$\tilde A$ $\tilde J$-symplectic and is ortho-normal with respect to the $X$-norm if and only if $(\tilde A^+)^T$ is $\bar J$-symplectic and is ortho-normal with respect to the $X^{-1}$-norm, where $\bar J = X^{-1/2}J^TX^{-1/2}$. Also we have that $\tilde A = \left( \left( \left(\tilde A^+\right)^T \right)^+ \right)^T$.

\emph{proof. }First we show that $(\tilde A^+)^T$ is $\bar J$-symplectic
\begin{equation}
\begin{aligned}
	\tilde A^+ \bar J (\tilde A^+)^T &= J^T \tilde A^T\underbrace{ \tilde J \bar J \tilde J ^T }_{\tilde J} \tilde A J \\
	&= J^T \underbrace{ \tilde A^T \tilde J \tilde A}_{J} J \\
	&= J.
\end{aligned}
\end{equation}
Now we can show that it is orthonormal with respect to the $X^{-1}$-norm
\begin{equation}
\begin{aligned}
	\tilde A^+ X^{-1} (\tilde A^+)^T &= J^T \tilde A^T \underbrace{ \tilde J X^{-1} \tilde J ^T}_{X} \tilde A J \\
	&= J^T \underbrace{ \tilde A^T X \tilde A }_{I} J \\
	&= I.
\end{aligned}
\end{equation}
The inverse statement can be shown similarly. One can easily check that the last statement of the lemma is also true.

The lemma above shows that if we enrich the basis $\tilde A$ such that $\tilde A$ remains ortho-normal with respect to $X$, then $(\tilde A ^+)^T$ remains orthonormal with respect to $X^{-1}$, and vise versa. This provides a recipe for ensuring $V = X (A^{\times})^T$. Assume that vector $z$ is in span$\left( X( A^{\times})^T\right)$. Then $X^{-1/2}z \in \text{span}\left(X^{1/2} (A^{\times})^T\right) = \text{span}\left( (\tilde A^+)^T\right)$.

We summarize the SDEIM method in Algorithm \ref{alg:2}.


\begin{algorithm} 
\caption{Symplectic Discrete Empirical Interpolation Method} \label{alg:2}
{\bf Input:} Symplectic basis $\tilde A=[  e_1,\dots,e_k,f_1,\dots,f_k ]$, nonlinear snapshots $\tilde S = X^{-1/2}\{ g( x_i ) \}$ and tolerance $\delta$
\begin{enumerate}
\item Compute $(\tilde A^+)^T = \{ e'_1,\dots,e'_k,f'_1,\dots,f'_k \}$
\item $i \leftarrow 1$
\item \textbf{while} max$\| \tilde s - (\tilde A^+)^T ((\tilde A^+)^T)^+\tilde s \|_2 > \delta$ for all $\tilde s\in \tilde S$
\item \hspace{0.5cm} $\tilde s^* \leftarrow \underset{\tilde s \in \tilde S}{\text{argmax }}\| \tilde s - (\tilde A^+)^T ((\tilde A^+)^T)^+ \tilde s \|_2$
\item \hspace{0.5cm} $\bar J$-orthogonalize $ \tilde S^*$ to obtain $e'_{k+1}$ 
\item \hspace{0.5cm} $f'_{k+i} = \bar J^{\times} e'_{k+i}$
\item \hspace{0.5cm} $(\tilde A^+)^T \leftarrow [e'_1,\dots,e'_{k+i},f'_1,\dots,f'_{k+i}]$
\item \hspace{0.5cm} $i\leftarrow i+1$
\item \textbf{end while}
\item compute $(((A^+)^T)^+)^T$
\end{enumerate}
\vspace{0.5cm}
{\bf Output:} Symplectic basis $\tilde A$ enriched with nonlinear snapshots.
\end{algorithm}





\subsection{Geometric integration of the Poisson system \eqref{eq:reduced-poisson} }
\Cref{eq:reduced-poisson} is a a non-canonical Hamiltonian system with constant non-canonical \todo[prepend]{is the matrix invertible?}invertible skew-symmetric structure matrix $\mathbb{J} := \tilde A ^+ \tilde J (\tilde A ^+)^T$ and symmetric matrix $A^T L A$ (see todo above). That it is a symplectic system can be seen from the following computation. The variational form associated with the equation is
\begin{align*}
\dot dy &= \mathbb{J} A^T L A dy, \\
\implies \dot dy \wedge \mathbb{J}dy &= 0, \\
\implies \frac{d}{dt} dy \wedge \mathbb{J}dy &= 0.
\end{align*}
Here we have used symmetry and skew symmetry of the involved matrices to get the penultimate equation. Symplectic ensures preservation of phase space volume among other things. One can now use a symplectic integrator of choice to preserve the symplectic structure. e.g. midpoint rule gives
\begin{align} \label{eq:imp}
D_t y = \mathbb{J} A^T L A A_t y
\end{align}
where $D_t$ and $A_t$ are standard forward difference and averaging operators. The associated variational equation is
\begin{align*}
D_t y &= \mathbb{J} A^T L A A_t y,\\
\implies D_t dy &= \mathbb{J} A^T L A A_t dy, \\
\implies D_t dy \wedge \mathbb{J} A_t dy &= \mathbb{J} A^T L A A_t dy \wedge \mathbb{J} A^T L A A_t dy, \\
\implies dy^{n+1} \wedge \mathbb{J} dy^{n+1} &= dy^{n} \wedge \mathbb{J} dy^{n}.
\end{align*}
Here we have again used the symmetry and skew-symmetry of the involved matrices and superscript $n$ is the discrete time index. Therefore the implicit midpoint rule \eqref{eq:imp} is a symplectic integrator. It is probably worth remarking that we have only used the skew-symmetry of the non-canonical structure matrix $\mathbb{J}$ and no other property of the matrix was needed.

In the theory of symplectic MOR, it is often convenient to reduce a canonical Hamiltonian system. In order to transform $\mathbb{J}$ into a canonical matrix, consider the following
\begin{align*}
\mathbb{J} = \tilde A ^+ \tilde J (\tilde A ^+)^T = J A^T X J XJXJX^TAJ.
\end{align*}
\emph{Assuming} $A$ and $X$ are block diagonal matrices, the matrix $\mathbb{J}$ is anti-block diagonal with zero diagonal blocks i.e. it has the following form
\begin{align*}
\mathbb{J} = \begin{bmatrix}
0 & V^T \\ -V & 0
\end{bmatrix}
\end{align*}
Here $V$ is a square matrix. Then there exists a congruent transformation $z = Qy$ with
\begin{align*}
Q = \begin{bmatrix}
I & 0 \\ 0 & V^{-1}
\end{bmatrix}
\end{align*}
such that
\begin{align*}
Q\mathbb{J}Q^T = \begin{bmatrix}
0 & I \\ -I & 0
\end{bmatrix} =: \mathbb{J}_{2k}
\end{align*}
where $k$ is the reduced dimension. With this congruent transformation, the non-canonical system \eqref{eq:reduced-poisson} turns into the following canonical Hamiltonian system
\begin{align}
\dot z = \mathbb{J}_{2k} A^TLAQ^{-1} z.
\end{align}
More about congruent transformation $Q$ can be found at e.g. \cite{Peng2016,Eves1980}.

\todo[inline]{This congruent transformation doesn't work because $A$ already satisfies symplecticness constraint.}

\bibliographystyle{abbrv}
\bibliography{main}

\end{document}
