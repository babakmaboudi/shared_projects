\documentclass[12pt,a4paper,draft]{article}
\usepackage[utf8]{inputenc}
\usepackage{amsmath}
\usepackage{amsfonts}
\usepackage{amssymb}
\author{Ashish Bhatt}

\newcommand{\dt}{\Delta t}

\title{Geometric MOR of Mechanical Systems}
\begin{document}
\maketitle

\section{Mechanical systems}
\subsection{Continuous system}
Consider the linear mechanical system
\begin{align}
M(\mu)\ddot{q} + D(\mu) \dot{q} + K(\mu) q = 0
\end{align}
where $M, D, K$ are parameter $\mu$ dependent matrices and $q$ is the unknown. We can write this as a first order system by introducing momentum $p$ as below
\begin{gather}
\begin{aligned}
\dot{q} &= M^{-1}p, \\
\dot{p} &= -DM^{-1}p - Kq.
\end{aligned}
\end{gather}
Due to the presence of damping in the system, it is not Hamiltonian but can be turned into one by enlarging the phase space. This attains the following objectives:
\begin{enumerate}
\item We can use geometric MOR techniques on the enlarged system.
\item The Hamiltonian can work as a Lyapunov function and hence can be used to derive and implement new error estimators for this mechanical system.
\end{enumerate}

Keeping this in mind, let us define an enlarged phase space
\begin{align}
U = \begin{bmatrix}
u(t) \\ \theta(s,t) \\ \phi(s,t)
\end{bmatrix}
\end{align}
where $u = [q^T, p^T]^T$. Over this phase space, we define Hamiltonian $\mathcal{H}$ as below
\begin{align}
\mathcal{H}(U) &= \frac{1}{2} \|\mathcal{K} U\|^2 \\
&= \frac{1}{2} \|Su-T\phi\|^2 + \frac{1}{2} q^T K q + \frac{1}{2} \int_{-\infty}^{\infty} \{ \|\theta(s,t)\|^2 + \|\partial_s \phi(s,t)\|^2 \} ds
\end{align}
where
\begin{gather}
\begin{aligned}
\mathcal{K} = \begin{bmatrix}
S & 0 & -T \\
0 & 1 & 0 \\
0 & 0 & \partial_s
\end{bmatrix}, S = \begin{bmatrix}
0 & 0 \\ 0 & I
\end{bmatrix}, \quad
T\phi = \int_{-\infty}^{\infty} \zeta(s) \phi(s) ds,\\
\zeta(s) = \frac{1}{2\pi} \int_{-\infty}^{\infty} \cos(\omega s) \sqrt{2\omega Im \hat{\chi}(\omega)} d\omega, \quad \chi(\tau) = \begin{bmatrix}
DM^{-1} & 0 \\ 0 & 0
\end{bmatrix}
\end{aligned}
\end{gather}
The equations of motion corresponding to this Hamiltonian in the enlarged space read
\begin{align}
\partial_t U(s,t) = \mathcal{J} \nabla_U \mathcal{H} = \mathcal{J} \mathcal{K}^T \mathcal{K} U(s,t)
\end{align}
Here $\nabla_U \mathcal{H}$ denotes the vector of variational derivatives and
\begin{align}
\mathcal{J} = \begin{bmatrix}
J & 0 & 0 \\
0 & 0 & -1 \\
0 & 1 & -1
\end{bmatrix}, J = \begin{bmatrix}
0 & I \\
-I & 0
\end{bmatrix}
\end{align}

\subsection{Discrete system}
(Symplectic) implicit midpoint rule is:
\begin{align}
\frac{U^{n+1,i+1/2} - U^{n,i+1/2}}{\dt} = \mathcal{J} \nabla_U \mathcal{H}(U^{n+1/2,i+1/2}).
\end{align}
Here $n$ and $i$ are the time and spatial indices, respectively, and $n+1/2$ or $i+1/2$ denote corresponding midpoints.

Hamiltonian-preserving discrete gradient method reads
\begin{align}
\frac{U^{n+1,i+1/2} - U^{n,i+1/2}}{\dt} = \mathcal{J} \bar{\nabla}_U \mathcal{H}(U^{n+1/2,i+1/2}).
\end{align}
where $\bar{\nabla}$ is a discrete gradient.

\subsection{Lyapunov function}
Since $\mathcal{H}$ satisfies 
$$\mathcal{H}(0) = 0, \mathcal{H}(U) > 0 \text{ for } U \neq 0, \text{ and } \frac{d}{dt}\mathcal{H} =0,$$
it is a good candidate for the Lyapunov function of the extended system.

\end{document}